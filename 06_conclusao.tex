% ---
% Conclusão
% ---
\chapter{Conclusão}
\label{c_conclusao}

Quais problemas este trabalho trata

O que foi feito até o momento (cap 2, 3 e 4)

Quais os próximos passos (com base no cap 4)

Resultados esperados... no modelo Univali está assim:

Este capítulo deve apresentar uma síntese sobre o trabalho desenvolvido, realizando uma análise a respeito do cumprimento dos objetivos estabelecidos e da verificação da hipótese de pesquisa inicial. Cada objetivo deve ser analisado, identificando-se o grau de atendimento (parcial ou integral), os problemas encontrados e as soluções adotadas, e justificando o porquê do não cumprimento integral (quando for o caso). Não devem ser apresentadas justificativas baseadas em dificuldades de natureza pessoal (ex. falta de tempo).


%_________________________
\section{Contribuição da Dissertação}
\label{c_conclusao-contribuicao}

Nesta seção, devem ser destacadas as principais contribuições do trabalho. Deve se identificar a relevância técnico-científica da pesquisa realizada, assim como os seus impactos social, ambiental e econômico (quando aplicável). Principalmente, deve-se ressaltar a contribuição do trabalho em relação ao estado da arte. Também podem ser identificados resultados alcançados quanto à publicações e patentes depositadas.


%_________________________
\section{Trabalhos Futuros}
\label{c_conclusao-trabalhos-futuros}

Esta seção deve identificar possíveis trabalhos que possam ser realizados a partir do desdobramento da pesquisa feita na dissertação. Procure discutir esses trabalhos como oportunidades de pesquisa que possam ser aproveitadas tanto por você como por outras pessoas.

Caso queira listar essas oportunidades, anteceda a lista por um parágrafo introdutório, como, por exemplo: “Ao longo do desenvolvimento deste trabalho, puderam ser identificadas algumas possibilidades de melhoria e de continuação a partir de futuras pesquisas, as quais incluem:”. Depois do parágrafo inicial, você pode listar as melhorias e continuações que podem ser feitas a partir do trabalho desenvolvido, mas procure comentar um pouco sobre cada proposta, mostrando que você já saberia como começar aquela nova pesquisa.

