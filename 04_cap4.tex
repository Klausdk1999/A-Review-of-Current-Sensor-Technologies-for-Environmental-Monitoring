% ----------------------------------------------------------------------- %
% Arquivo: cap4.tex
% ----------------------------------------------------------------------- %
\chapter{4  DESENVOLVIMENTO}
\label{c_cap4}

Neste capítulo, deve-se apresentar a contribuição da dissertação, ou seja, a solução desenvolvida para tratar o problema de pesquisa que motivou o trabalho. O título do capítulo é livre, podendo ser utilizado “Desenvolvimento”, ou outro que permita identificar mais claramente a solução desenvolvida.

Havendo necessidade, e em comum acordo com o orientador, este capítulo pode ser desdobrado em dois ou mais capítulos, se isso contribuir para melhorar a organização do documento. Por exemplo, se para um mesmo problema foram desenvolvidas duas soluções diferentes e, se para cada uma delas se justificar um capítulo exclusivo, existe liberdade para dividir este capítulo. O(s) capítulo(s) pode(m) ser organizado(s) de diferentes maneiras, ficando a cargo do autor e do seu orientador identificar aquela mais adequada à natureza do trabalho realizado. 

Em casos em que a dissertação possui uma contribuição complementar com um volume expressivo de informações que não são necessárias ao corpo da dissertação, essas informações podem ser colocadas em apêndices ou publicadas em um relatório técnico do Curso. Por exemplo, nos casos em que são desenvolvidos softwares que possuam manual de usuário, recomenda-se a sua publicação como relatório técnico, o qual pode ser citado pela dissertação. O mesmo se aplica ao projeto detalhado de sistemas de software e/ou de hardware desenvolvidos.



%_________________________
%_________________________
\section{Visão Geral e Premissas}
\label{s_c4_visao_geral}





%_________________________
\subsection{Detalhamento da Solução}
\label{ss_c4_detalhamento_mGov-BR}




%_________________________
\subsection{Requisitos Funcionais e Não Funcionais do Sistema Proposto}
\label{ss_c4_RF_RNF}

\begin{itemize}
    \item RF 01: requisito funcional 1;
    \item RF 02: requisito funcional 2;
    \begin{itemize}
        \item subitem do requisito funcional 2;        
    \end{itemize}    
\end{itemize}


\begin{itemize}
    \item RNF 01: requisito não funcional 1;
    \item RNF 02: requisito não funcional 2;
\end{itemize}


%_________________________
\subsection{Cenários de Uso}
\label{ss_c4_cenario-uso}

Para exemplificar a utilização da sistema XYZ proposto...


%_________________________
\subsubsection{Caso 01}
\label{sss_c4_caso-01}

Descrição do caso de uso 01;



%_________________________
\subsubsection{Caso 02}
\label{sss_c4_caso-02}

Descrição do caso de uso 02;



%_________________________
%_________________________
\section{Protótipo Desenvolvido}
\label{s_c4_prototipo}

Com o objetivo de avaliar a funcionalidade, usabilidade e desempenho da solução proposta, foi construído um protótipo em software, constituído de tal e tal coisa. Os atores são:

\begin{enumerate}
    \item Ator 1;
    \item Ator 2; e,
    \item Ator 3.
\end{enumerate}

Esta seção apresenta as ferramentas e tecnologias adotadas, bem como o detalhamento técnico da implementação do protótipo e a integração entre os atores.



%_________________________
\subsection{Ferramentas e Tecnologias Selecionadas}
\label{ss_c4_tecnologias}

Tecnologias seleciondas...




%_________________________
\subsection{Aplicativo X}
\label{ss_c4_aplicativoX}

Descrição do aplicativo X;



%_________________________
\section{Considerações}
\label{s_c4_consideracoes}

Este capítulo pode ter uma última seção como esta denominada ``Considerações'' ou ``Discussão'' discutindo o trabalho desenvolvido e fazendo uma ligação com o capítulo de análise de resultados.