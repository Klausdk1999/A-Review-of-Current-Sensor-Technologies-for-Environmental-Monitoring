% ----------------------------------------------------------------------- %
% Arquivo: introducao.tex
% ----------------------------------------------------------------------- %


% ----------------------------------------------------------------------- %
\chapter{Introdução}
\label{c_introducao}


%Exemplo de texto com citação \cite{onu:14}.
%Segundo \citeonline{onu:14}, aqui é uma citação direta.

%O primeiro capítulo apresenta o trabalho, identificando-o para o leitor. O objetivo é estabelecer uma introdução ao assunto, definir o problema de pesquisa; apresentar, delimitar e justificar a solução proposta; apresentar os objetivos da dissertação (geral e específicos); caracterizar e descrever a metodologia adotada, e descrever a estrutura da dissertação. Este capítulo é importante para que o leitor tenha uma visão clara do conteúdo do texto e o que você fez.  Deve-se usar o tempo verbal presente, por exemplo: “O presente trabalho apresenta uma nova abordagem para ...”

%A primeira parte da Introdução é a contextualização do trabalho, a qual deve iniciar diretamente a partir do título do capítulo. Esta seção deve possuir referências bibliográficas (sempre buscando diferentes autores). É neste momento que você estará apresentando o seu trabalho e indicando o contexto em que ele se encontra. Você pode iniciar com uma visão mais abrangente e ir focalizando o contexto até o trabalho em si.

%Ao final desta seção, você irá ter algo como: “... Dentro deste contexto, este trabalho procura fazer uma contribuição na área de ....” (não é a definição do objetivo, mas uma delimitação do tema).

A transição energética atual exige um gerenciamento eficaz do fluxo de potência em usinas fotovoltaicas. Essas usinas são fontes de energia elétrica confiáveis e eficientes. Esta gestão torna-se ainda mais crítica quando tais instalações são integradas com sistemas de armazenamento de energia baseados em baterias. A energia solar é renovável, mas é intrinsecamente variável devido às flutuações climáticas e ao ciclo diário, o que pode resultar em oscilações na produção de energia. Além disso, adicionar sistemas de armazenamento de energia por baterias aumenta a complexidade, exigindo uma otimização cuidadosa para assegurar a utilização eficiente da energia armazenada e minimizar as perdas \cite{bueno2013}.

Os desafios principais associados ao armazenamento de energia residem em lidar com a variabilidade inerente à geração solar. Essa variabilidade pode desencadear flutuações rápidas e imprevisíveis no fluxo de potência. Esta realidade requer algoritmos que apoiem sistemas de gerenciamento avançados capazes de serem preditivos em alguns casos permitindo regular estas flutuações para manter a rede elétrica estável. Além disso, otimizar o aproveitamento da energia armazenada é outro desafio crítico. Isso exige tomadas de decisão em tempo real sobre quando armazenar, liberar ou injetar energia na rede. Estas decisões devem considerar fatores como custos energéticos, demanda do sistema e previsões meteorológicas \cite{castro2016}.

A crescente integração da geração distribuída, especialmente a fotovoltaica, torna o desenvolvimento de modelos de previsão um requisito essencial para facilitar a alta penetração de fontes renováveis em conjunto com fontes tradicionais, como a hidrelétrica. Esses modelos de previsão devem ser capazes de lidar com a natureza estocástica e a variabilidade das fontes renováveis para garantir a segurança e a confiabilidade da rede elétrica. A previsão da geração de energia fotovoltaica em sistemas de micro e minigeração apresenta desafios adicionais. As distribuidoras de energia elétrica precisam implementar um controle integrado dos fluxos de energia entre cargas e fontes. As simulações são ferramentas essenciais para testar e validar o desempenho de sistemas de gerenciamento de energia em condições controladas e variáveis. Isso permite identificar possíveis falhas e otimizar o sistema antes da implementação em larga escala \cite{bastos2020}.
% ----------------------------------------------------------------------- %
\section{Problema de Pesquisa}
\label{s_cintro_problema_pesquisa}

%Nesta seção, você deve descrever qual é o problema a ser resolvido. É necessário evidenciar que existem questões em aberto, que o tema é complexo e que há interesse na comunidade em resolver o problema. O texto deve responder às seguintes perguntas:%

%\begin{itemize}
   % \item Qual a relevância e complexidade do problema apresentado?
    
    %\item Existe alguma solução consolidada ou o problema ainda está em aberto?
%\end{itemize}

%Nesta seção, você deve ainda indicar quais as perguntas de pesquisa que você buscou responder por meio do seu trabalho. Usualmente, as perguntas permitem a formulação de uma ou mais hipóteses que serão apresentadas na seção a seguir (Solução Proposta).

%No final do texto, coloca-se as perguntas de pesquisa:

%\begin{enumerate}
 %   \item Pergunta 1?
    
  %  \item Pergunta 2? 
%\end{enumerate}

O aumento contínuo da demanda por energia elétrica apresenta um desafio significativo para os sistemas de energia atuais. Em particular, a integração eficaz de fontes de energia renovável, como a energia solar fotovoltaica, com sistemas de armazenamento de energia é uma questão complexa e crucial para garantir um suprimento confiável e sustentável de energia. 

A relevância desse problema é evidenciada pela necessidade de reduzir as emissões de gases de efeito estufa e mitigar os impactos das mudanças climáticas. Há outros benefícios socioambientais no uso de baterias como por exemplo, reduzem
a necessidade e impactos da instalação de linhas de transmissão e distribuição, e permitem que comunidades dispersas não conectadas à rede elétrica tenham acesso à energia \cite{wef2019}.

Embora existam soluções consolidadas para algumas partes do problema, como a tecnologia de baterias para armazenamento de energia, há várias questões em aberto que tornam o tema complexo. Por exemplo, a otimização do sistema de armazenamento de energia para atender às necessidades específicas de um sistema com alta demanda de energia, combinada com a variabilidade da geração de energia solar, ainda é um desafio. Além disso, questões relacionadas à integração do sistema com a rede elétrica convencional e a maximização da eficiência energética também estão em foco.

As perguntas de pesquisa que orientam este trabalho são:

\begin{enumerate}
   \item Como a combinação de armazenamento de energia por baterias e geração de energia solar fotovoltaica pode ser otimizada para atender a uma alta curva de demanda de energia?
    
   \item Quais são os principais desafios na integração de sistemas de armazenamento de energia e geração de energia renovável e como esses desafios podem ser superados para promover uma transição energética mais eficiente?
\end{enumerate}



% ----------------------------------------------------------------------- %
\subsection{Solução Proposta}
\label{ss_cintro_solucao}

Nesta subseção, você deve apresentar a sua proposta de solução para o problema identificado. Veja que a solução não precisa resolver todo o problema de pesquisa, mas precisa indicar que será uma contribuição (a justificativa detalhada estará na próxima seção).

Além disso, você deve apresentar as suas hipóteses de pesquisa. A hipótese é uma afirmação que você faz no início e busca avaliar ao final do trabalho, demonstrando como foi todo o processo para essa avaliação, seguindo o método científico.

No final da seção ficam as hipóteses:

\begin{itemize}
    \item[H1 -] Primeira hipótese.
    
    \item[H2 -] Segunda hipótese.
    
    \item[H3 -] Terceira hipótese.
\end{itemize}





% ----------------------------------------------------------------------- %
\subsection{Delimitação de Escopo}
\label{ss_cintro_escopo}

Nesta subseção, você deve estabelecer os limites do trabalho, deixando claro para o leitor o escopo da pesquisa realizada. Você deve identificar aquilo que será feito e aquilo que não será feito, ou seja, as limitações do trabalho. Procure ser o mais honesto possível. Evite criar expectativas que ultrapassem a capacidade do trabalho.



% ----------------------------------------------------------------------- %
\subsection{Justificativa}
\label{ss_cintro_justificativa}

Aqui, o foco está em justificar a solução proposta. Você deve deixar muito claro para o leitor qual é a efetiva contribuição do seu trabalho, procurando responder a perguntas:

\begin{itemize}
    \item Qual é a relevância da solução da proposta?
    
    \item Qual é a complexidade da solução proposta?
    
    \item Qual é a aplicabilidade da solução?
    
    \item A solução é viável?
    
    \item Qual é o seu diferencial a outros similares?
    
    \item Qual é o problema que ele irá resolver?
    
    \item Qual é a motivação para ele?
\end{itemize}

Procure utilizar referências bibliográficas para ajudar na defesa da relevância da solução proposta.

A justificativa, como o próprio nome indica, é a argumentação a favor da validade da realização do trabalho proposto, identificando as contribuições esperadas e o diferencial relação aos trabalhos similares.

No final, o presente trabalho se justifica cientificamente por ...



% ----------------------------------------------------------------------- %
\section{Objetivos}
\label{s_cintro_objetivos}

Esta seção formaliza os objetivos do trabalho previamente definidos no Projeto de Dissertação e eventualmente revisados a posteriori. O cumprimento desses objetivos deve ser avaliado no capítulo final da dissertação.


% ----------------------------------------------------------------------- %
\subsection{Objetivo Geral}
\label{ss_cintro_objetivo_geral}

Procure utilizar apenas uma única frase para descrever o objetivo geral, iniciando com um verbo no infinitivo. Evite muitos conectores e explicações, pois eles não fazem parte do objetivo geral e já constituem parte dos objetivos específicos.




% ----------------------------------------------------------------------- %
\subsection{Objetivos Específicos}
\label{ss_cintro_objetivos_espec}

\begin{enumerate}
    \item Esta seção é uma lista de itens (como esta), cada um sendo um objetivo. É interessante que esses objetivos sejam numerados de alguma forma (o propósito desta numeração não é criar uma ordem de importância, mas permitir que o objetivo possa ser referenciado ao longo do projeto);
    
    \item Deve se indicar todas as metas do trabalho. As perguntas a serem respondidas são “onde você quer chegar com este trabalho?”, “o que deve ser gerado após a conclusão do trabalho?”;
    
    \item Procure ser realista e não escreva objetivos muito gerais ou muito abertos;
    
    \item Evite listar muitos objetivos específicos e defina objetivos que sejam viáveis dentro do prazo que você terá para a execução do seu trabalho;
    
    \item Evite colocar como objetivos específicos “O estudo ou aprofundamento de alguma coisa”. O estudo é um meio para alcançar o seu objetivo (a não ser que o seu objetivo seja apenas o estudo daquela alguma coisa - o que, usualmente, não deverá ser aceito como um trabalho deste porte);
    
    \item Você deve evitar o preenchimento de uma seqüência de atividades realizadas (ver metodologia). Essa seqüência de atividades é o plano de trabalho e mostra como você trabalhou para alcançar os objetivos definidos aqui. O plano de trabalho é apresentado no Projeto de Dissertação e no Exame de Qualificação, nunca no texto final da dissertação;
    
    \item Evite objetivos pessoais e procure focar em objetivos do trabalho;
    
    \item Cada um dos objetivos específicos deverá ser trabalhado mais tarde nas conclusões da Dissertação, pois será preciso indicar como estes objetivos foram alcançados e, caso contrário, justificar o porquê do não atendimento a um objetivo traçado no início da pesquisa.    
\end{enumerate}



O texto fica mais ou menos com o seguinte conteúdo:

\begin{enumerate}
    \item Primeiro objetivo específico;
    
    \item Segundo objetivo específico; e

    \item Terceiro objetivo específico.
\end{enumerate}




% ----------------------------------------------------------------------- %
\section{Metodologia}
\label{s_cintro_metodologia}

Nesta seção, deve-se classificar a metodologia utilizada na pesquisa e apresentar uma síntese dos procedimentos metodológicos utilizados para o desenvolvimento da dissertação. É recomendável dividir esta seção nas subseções apresentadas a seguir. 


% ----------------------------------------------------------------------- %
\subsection{Metodologia da Pesquisa}
\label{ss_cintro_metod_pesquisa}

Esta seção classifica a metodologia de pesquisa utilizada. Antes de elaborá-la, você deve ler livros e artigos sobre Metodologia Científica, incluindo o manual “Elaboração de trabalhos acadêmico-científicos” disponibilizado pela UNIVALI na página da Biblioteca. 
Estabeleça a definição de método, relacionando-o com seu trabalho. Identifique e justifique o tipo de método adotado no trabalho:

\begin{itemize}   
    \item Método indutivo;
    
    \item Método dedutivo;

    \item Método hipotético-dedutivo; ou
    
    \item Outros métodos
\end{itemize}

Caracterize a pesquisa no seu trabalho e justifique sob os diferentes pontos de vista da metodologia científica. 

\textbf{Sob o ponto de vista de sua natureza:} 
\begin{itemize}
    \item Pesquisa básica; ou

    \item Pesquisa aplicada.
\end{itemize}

\textbf{Sob o ponto de vista da forma de abordagem do problema:}
\begin{itemize}
    \item Pesquisa quantitativa; ou
    
    \item Pesquisa qualitativa.
\end{itemize}

\textbf{Sob o ponto de vista de seus objetivos:}
\begin{itemize}
    \item Pesquisa exploratória;
    
    \item Pesquisa descritiva; ou
    
    \item Pesquisa explicativa.
\end{itemize}
Lembre-se que você está falando de um trabalho específico (o seu) e, portanto, você deve indicar por que seu trabalho é classificado de um jeito e não de outro. Veja também que, eventualmente, sob um determinado ponto de vista, o trabalho pode se enquadrar em mais de um tipo de pesquisa. Neste caso, cada uma deve ser justificada.



% ----------------------------------------------------------------------- %
\subsection{Procedimentos Metodológicos}
\label{ss_cintro_proced_metodologicos}

Esta seção deve apresentar como o trabalho foi desenvolvido para atingir os seus objetivos. O texto deve demonstrar de modo claro e objetivo o caminho utilizado para construir a solução proposta.

Você deve identificar os procedimentos técnicos que você utilizou, como, por exemplo:

\begin{itemize}
    \item Pesquisa bibliográfica;
    \item Pesquisa documental;
    \item Pesquisa experimental;
    \item Levantamento;
    \item Estudo de caso;
    \item Pesquisa ex post facto;
    \item Pesquisa-ação; 
    \item Pesquisa participante; ou
    \item Outros.
\end{itemize}

Você deve definir as etapas utilizadas na execução do seu trabalho, por meio de um plano de trabalho descrito textualmente, explicando as atividades realizadas, os resultados obtidos, os artefatos desenvolvidos, etc. Você deve explorar os procedimentos técnicos comentados na seção anterior. Você deve fazer uma ligação entre as etapas executadas na sua pesquisa e os objetivos específicos da dissertação. Todos os objetivos específicos devem ser atendidos com a execução dos itens do plano de trabalho.



% ----------------------------------------------------------------------- %
\section{Estrutura da Dissertação}
\label{s_cintro_estrutura}

Nesta seção, deve-se descrever a estrutura do texto, de forma textual, identificando o conteúdo e as contribuições de cada capítulo da dissertação. Abaixo, segue um exemplo de redação a ser utilizada.

O trabalho está organizado em N capítulos correlacionados. O Capítulo 1, Introdução, apresentou por meio de sua contextualização o tema proposto neste trabalho. Da mesma forma foram estabelecidos os resultados esperados por meio da definição de seus objetivos e apresentadas as limitações do trabalho permitindo uma visão clara do escopo proposto.

O Capítulo 2 apresenta a fundamentação teórica ...

O Capítulo 3 apresenta o estado da arte sobre ..., permitindo que ...

O Capítulo 4 apresenta ...

O Capítulo 5 apresenta ....

No Capítulo N, são tecidas as conclusões do trabalho, relacionando os objetivos identificados inicialmente com os resultados alcançados. São ainda propostas possibilidades de continuação da pesquisa desenvolvida a partir das experiências adquiridas com a execução do trabalho.
