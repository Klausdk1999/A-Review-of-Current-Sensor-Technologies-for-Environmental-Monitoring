% ------------------------------------------------------------------------
% ------------------------------------------------------------------------
% Modelo de TCC do curso de Engenharia de Controle e Automação - UFSC/Campus Blumenau
% Autor: Ciro André Pitz
% Revisão: Brenda Teresa Porto de Matos
% O presente modelo foi obtido a partir do modelo desenvolvido por Alisson Lopes Furlani disponível na BU/UFSC.
% ------------------------------------------------------------------------
% ------------------------------------------------------------------------

\documentclass[
12pt,				% tamanho da fonte
%openright,			% capítulos começam em pág ímpar (insere página vazia caso preciso)
oneside,			% para impressão no anverso. Oposto a twoside
a4paper,			% tamanho do papel. 
chapter=TITLE,		% títulos de capítulos convertidos em letras maiúsculas
section=TITLE,		% títulos de seções convertidos em letras maiúsculas
%subsection=TITLE,	% títulos de subseções convertidos em letras maiúsculas
%subsubsection=TITLE,% títulos de subsubseções convertidos em letras maiúsculas
% -- opções do pacote babel --
english,			% idioma adicional para hifenização
brazil				% o último idioma é o principal do documento
]{abntex2}

\usepackage{configs/eca_ufsc_bnu} % personalização da ABNTEX2 
\addbibresource{pos_textual/referencias.bib} % Seus arquivos de referências

%---------------------------------------------------------------------------------------------
%--------- DADOS BÁSICOS DO TCC (Preencher todos) -------------------------------------
%---------------------------------------------------------------------------------------------
%Substituir 'Nome completo do autor' pelo seu nome.
\autor{Klaus Dieter Kupper}
% FIXME Substituir 'Título do trabalho' pelo título da trabalho.
\titulo{A Review of Current Sensor Technologies for Environmental Monitoring }
% Substituir 'Subtítulo (se houver)' pelo subtítulo da trabalho. 
% Se não houver subtítulo, basta deletar o texto. // the Era of IoT and Wireless Sensor Networks
\subtitulo{For the Era of IoT and Wireless Sensor Networks}
% Substituir 'Orientador' pelo nome do seu orientador.
\orientador{Prof. Dr. Jordan Sausen}
% Se for orientado por uma mulher, comente a linha acima e descomente a linha a seguir.
% \orientador[Orientadora]{Orientadora, Dra.}
% Substituir 'XXXXXX' pelo nome do seu  coorientador. Caso não tenha coorientador, comente a linha a seguir.
%\coorientador{Prof. Coorientador, Dr.}
% Se for coorientado por uma mulher, comente a linha acima e descomente a linha a seguir.
% \coorientador[Coorientadora]{Coorientadora, Dra.}
\dia{10}
\mes{Julho}
\ano{2025}
\local{Itajaí}
\formacao{Mestrado em Computação Aplicada}
% Se for mulher, comente a linha acima e descomente a linha a seguir.
%\formacao{Engenheira de Controle e Automação}
\bancaa{Prof. Dr. Carlos Roberto Moratelli} %Primeiro membro da banca (normalmente o orientador).
\bancab{Prof. Dr. Jordan Sausen} %Segundo membro da banca
\bancac{Prof. Dr. Ciro André Pitz} %Terceiro membro da banca

% Resumo e palavras-chave do trabalho
\resumotcc{
Este trabalho apresenta uma análise comparativa de sensores ultrassônicos, LiDAR e câmeras para a medição do nível de rios. O objetivo é avaliar o desempenho desses sensores em diferentes condições ambientais, considerando fatores como precisão, alcance máximo e limitações operacionais. Os testes experimentais serão conduzidos em um ambiente controlado e em um cenário real, permitindo uma avaliação detalhada das capacidades de cada tecnologia. Com base nos resultados obtidos, serão indicadas as aplicações mais adequadas para cada tipo de sensor e determinada a melhor opção para o monitoramento do nível de rios. Os achados deste estudo podem contribuir para o aprimoramento de sistemas de monitoramento hidrológico, auxiliando na escolha de sensores mais eficientes e adequados para diferentes cenários.
}
\palavraschave{Sensores Ultrassônicos; LiDAR; Câmeras; Monitoramento Hidrológico; Sensoriamento Remoto.}

\abstracttcc{This work presents a comparative analysis of ultrasonic sensors, LiDAR, and cameras for river level measurement. The objective is to evaluate the performance of these sensors under different environmental conditions, considering factors such as accuracy, maximum range, and operational limitations. Experimental tests will be conducted in a controlled environment and a real-world scenario, enabling a detailed assessment of each technology's capabilities. Based on the obtained results, the most suitable applications for each sensor type will be identified, and the best option for river level monitoring will be determined. The findings of this study may contribute to the improvement of hydrological monitoring systems, helping to select more efficient and appropriate sensors for different scenarios.
}
\keywords{Ultrasonic Sensors; LiDAR; Cameras; Hydrological Monitoring; Remote Sensing.}

%Agradecimentos (opcional). Caso não queira inserir, deixe em branco (\agradecimentostcc{} )
\agradecimentostcc{Gostaria de expressar minha profunda gratidão ao meu professor orientador, pela sua orientação acadêmica e apoio ao longo deste trabalho. Suas orientações sábias e paciência foram fundamentais para o seu desenvolvimento.

A todos os professores que tive ao longo da minha jornada acadêmica, sou imensamente grato. Cada aula, conselho e palavra contribuíram para o meu crescimento pessoal e profissional. Em especial, agradeço por terem despertado em mim o interesse pela engenharia e tecnologia, que hoje são a minha paixão e a área em que atuo.

Agradeço também aos meus pais, cujo amor, apoio incondicional e crença em mim foram essenciais para que eu chegasse até aqui. Vocês foram minha inspiração e força motriz em todos os momentos desafiadores.

Aos meus colegas, expresso minha gratidão pelo apoio, compreensão e companheirismo ao longo dessa jornada. Nossos momentos de estudo, discussões e desafios compartilhados foram fundamentais para o meu crescimento e formação.

Por fim, gostaria de estender meu agradecimento a todos que, de alguma forma, contribuíram para a minha formação. Cada gesto e palavra de apoio foram importantes para o meu crescimento como pessoa e profissional.}

%Epígrafe (opcional). Caso não queira inserir, deixe em branco (\epigrafetcc{} )
\epigrafetcc{"O computador é a bicicleta da mente." (Steve Jobs, 1985)}

%Decatória (opcional). Caso não queira inserir, deixe em branco (\dedicatoriatcc{} )
\dedicatoriatcc{Este trabalho é dedicado aos meus colegas de classe, a minha namorada e aos meus queridos pais.}

%Lista de quadros
%Além de figuras e tabelas, o TCC contém quadros? Caso afirmativo digite sim ou deixe em branco para não (\contemquadros{}).
\contemquadros{sim}

%Lista de siglas (opcional).
%Deseja incluir lista de abreviaturas e siglas? Caso afirmativo digite sim ou deixe em branco para não (\contemsiglas{}).
\contemsiglas{sim}

%Lista de símbolos (opcional).
%Deseja incluir lista de símbolos? Caso afirmativo digite sim ou deixe em branco para não (\contemsimbolos{}).
\contemsimbolos{sim}

%-------------------------FIM DOS DADOS BÁSICOS DO TCC--------------------------------------------

% ajusta espaçamento das listas itemize e enumerate
\setitemize{topsep=0pt,itemsep=0pt,leftmargin=\parindent+\labelwidth-\labelsep}
\setenumerate{topsep=0pt,itemsep=0pt,leftmargin=\parindent+\labelwidth-\labelsep}

% define a macro \Autoref to allow multiple references to be passed to \autoref
\makeatletter
\newcommand\Autoref[1]{\@first@ref#1,@}
\def\@throw@dot#1.#2@{#1}% discard everything after the dot
\def\@set@refname#1{%    % set \@refname to autoefname+s using \getrefbykeydefault
	\edef\@tmp{\getrefbykeydefault{#1}{anchor}{}}%
	\xdef\@tmp{\expandafter\@throw@dot\@tmp.@}%
	\ltx@IfUndefined{\@tmp autorefnameplural}%
	{\def\@refname{\@nameuse{\@tmp autorefname}s}}%
	{\def\@refname{\@nameuse{\@tmp autorefnameplural}}}%
}
\def\@first@ref#1,#2{%
	\ifx#2@\autoref{#1}\let\@nextref\@gobble% only one ref, revert to normal \autoref
	\else%
	\@set@refname{#1}%  set \@refname to autoref name
	\@refname~\ref{#1}% add autoefname and first reference
	\let\@nextref\@next@ref% push processing to \@next@ref
	\fi%
	\@nextref#2%
}
\def\@next@ref#1,#2{%
	\ifx#2@ e~\ref{#1}\let\@nextref\@gobble% at end: print e+\ref and stop
	\else, \ref{#1}% print  ,+\ref and continue
	\fi%
	\@nextref#2%
}
\makeatother

% Cria comando para referenciar Anexo automaticamente \refanexo
\newcommand{\refanexo}[1]{\hyperref[#1]{Anexo~\ref{#1}}}

% Define comandos para tabelas que permite ajustar o tamanho da coluna e manter alinhamento C, R ou L
%\newcommand{\PreserveBackslash}[1]{\let\temp=\\#1\let\\=\temp}
\newcolumntype{C}[1]{>{\centering\let\arraybackslash}m{#1}}
\newcolumntype{R}[1]{>{\RaggedLeft\let\arraybackslash}m{#1}}
\newcolumntype{L}[1]{>{\RaggedRight\let\arraybackslash}m{#1}}


% ---
% Filtering and Mapping Bibliographies
% ---
\DeclareSourcemap{
	\maps[datatype=bibtex]{
		% remove fields that are always useless
		\map{
			\step[fieldset=abstract, null]
			\step[fieldset=pagetotal, null]
			\step[fieldset=doi, null]
		}
		% remove URLs for types that are primarily printed
		\map{
			\pernottype{software}
			\pernottype{online}
			\pernottype{report}
			\pernottype{techreport}
			\pernottype{standard}
			\pernottype{manual}
			\pernottype{misc}
			\step[fieldset=url, null]
			\step[fieldset=urldate, null]
		}
		\map{
			\pertype{inproceedings}
			% remove mostly redundant conference information
			%\step[fieldset=venue, null]
			%\step[fieldset=eventdate, null]
			%\step[fieldset=eventtitle, null]
			% do not show ISBN for proceedings
			\step[fieldset=isbn, null]
			% Citavi bug
			%\step[fieldset=volume, null]
		}
	}
}
% ---

\preambulo
{%
	Trabalho de Conclusão de Curso de Mestrado em Computação Aplicada da Universidade do Vale do Itajaí como requisito para a obtenção~do~título~de~\imprimirformacao.
}
% ---

% ---
% Configurações de aparência do PDF final
% ---
% alterando o aspecto da cor azul
\definecolor{blue}{RGB}{41,5,195}
% informações do PDF
\makeatletter
\hypersetup{
	%pagebackref=true,
	pdftitle={\@title}, 
	pdfauthor={\@author},
	pdfsubject={\imprimirpreambulo},
	pdfcreator={LaTeX with abnTeX2},
	pdfkeywords={ufsc, latex, abntex2}, 
	colorlinks=true,       		% false: boxed links; true: colored links
	linkcolor=black,%blue,          	% color of internal links
	citecolor=black,%blue,        		% color of links to bibliography
	filecolor=black,%magenta,      		% color of file links
	urlcolor=blue,
	bookmarksdepth=4
}
\makeatother
% ---

% Definição das siglas e símbolos

%----------------- LISTA DE ABREVIATURAS E SIGLAS--------------------------------------
\siglalista{REST}{\textit{Representational State Transfer}}
\siglalista{ORM}{\textit{Object-Relational Mapping}}
\siglalista{UUID}{\textit{Universal Unique Identifier}}
\siglalista{API}{\textit{Application Programming Interface}}
\siglalista{QR}{\textit{Quick Response}}
\siglalista{CRUD}{\textit{Create, Read, Update, and Delete}}
\siglalista{tRPC}{\textit{Typescript Remote Procedure Call}}
\siglalista{HTTP}{\textit{Hypertext Transfer Protocol}}
\siglalista{IA}{Inteligência Artificial}
\siglalista{ERD}{\textit{Entity-Relationship Diagram}}
\siglalista{CORS}{\textit{Cross-origin Resource Sharing}}
\siglalista{WWW}{\textit{World Wide Web}}
\siglalista{HTML}{\textit{HyperText Markup Language}}
\siglalista{JWT}{\textit{JSON Web Token}}
\siglalista{JSON}{\textit{JavaScript Object Notation}}
\siglalista{SQL}{\textit{Structured Query Language}}
\siglalista{IoT}{Internet of Things}
\siglalista{LoRa}{Long Range}
\siglalista{LoRaWAN}{Long Range Wide Area Network}
\siglalista{LiDAR}{Light Detection and Ranging}
\siglalista{WSN}{Wireless Sensor Network}
\siglalista{ESP32}{Espressif Systems 32-bit Microcontroller}
\siglalista{UART}{Universal Asynchronous Receiver-Transmitter}
\siglalista{SPI}{Serial Peripheral Interface}
\siglalista{I2C}{Inter-Integrated Circuit}
\siglalista{MAC}{Medium Access Control}
\siglalista{ADR}{Adaptive Data Rate}
\siglalista{CSS}{Chirp Spread Spectrum}
\siglalista{RMSE}{Root Mean Square Error}
\siglalista{RADAR}{Radio Detection and Ranging}
\siglalista{TOF}{Time Of Flight}


%Para usar uma dada sigla ABC ao longo do texto, use \glsxtrfull{ABC} se quiser apresentar a sigla e sua definição.
%Se quiser apresentar apenas a sigla, use \gls{ABC}.

%-----------------SÍMBOLOS---------------------------------------------------------------
\simbololista{C}{\ensuremath{C}}{Circunferência de um círculo}
\simbololista{pi}{\ensuremath{\pi}}{Número pi} 
\simbololista{r}{\ensuremath{r}}{Raio de um círculo}
\simbololista{A}{\ensuremath{A}}{Área de um círculo}

%Para usar um dado símbolo SIMB ao longo do texto, use \gls{SIMB}.

% compila a lista de abreviaturas e siglas e a lista de símbolos
\makenoidxglossaries 

% compila o indice
\makeindex


% ------------------------------------------------------------------------------------------------
% --------------------------INÍCIO DO DOCUMENTO---------------------------------------------
% ------------------------------------------------------------------------------------------------
\begin{document}
	
	% Seleciona o idioma do documento (conforme pacotes do babel)
	%\selectlanguage{english}
	\selectlanguage{brazil}
	
	% Retira espaço extra obsoleto entre as frases.
	\frenchspacing 
	
	% Espaçamento 1.5 entre linhas
	\OnehalfSpacing
	
	% Corrige justificação
	%\sloppy
	

	%Elementos pré-textuais
	% \pretextual %a macro \pretextual é acionado automaticamente no início de \begin{document}
	% Capa, folha de rosto, ficha bibliografica, errata, folha de aprovação
	% Dedicatória, agradecimentos, epígrafe (opcional), resumos, listas
	\input{pre_textual/pretextual}

	% Elementos textuais
	\textual
	
	% 1 - Introdução
	% ----------------------------------------------------------
\chapter{Introdução} \label{cap:intro}
% ----------------------------------------------------------
% Motivation for environmental monitoring
% AI Increases the need for data, which can be supplied from sensors etc.
% Importance of Wireless Sensor Networks (WSNs) in environmental monitoring.
% Increasing relevance of low-power, long-range technologies like LoRa.
% Categorization: Water, Soil, and Air/Chemical monitoring.
% Smart cities?
% Outline of paper structure.


Floods, particularly flash floods, are among the most impactful natural disasters worldwide, causing substantial economic losses, infrastructure damage, and fatalities \cite{jonkman_2005_global, santos_2014_indstria}. These events are characterized by their sudden onset and rapid development, often occurring at the outlet of small catchments in response to intense localized rainfall, with response times of only a few hours \cite{borga_2014_hydrogeomorphic} . Compounding the issue, climate change is intensifying the global hydrological cycle, leading to an increase in the frequency and severity of extreme weather events, including floods. This reinforces the need for improved flood monitoring, risk management strategies and alert systems to safeguard communities \cite{hall_2014_understanding, bragana_2024_anlise}.

Flash floods and debris flows often escape conventional hydrometeorological monitoring due to their spatial and temporal unpredictability \cite{borga_2014_hydrogeomorphic,hall_2014_understanding}. Accurate and real-time data collection becomes essential to calibrate hydrological and hydrodynamic models such as rainfall–runoff systems, flood discharge and water supply volumes which serve as the foundation for effective early warning systems \cite{lin_2020_semantic, lo_2015_visual, iqbal_2021_how}. In addition to instrumental data, combining systematic flood observations with historical and documentary records has been highlighted as a valuable approach to better understand long-term flood regime dynamics, including the detection of flood-rich and flood-poor periods influenced by natural variability \cite{borga_2014_hydrogeomorphic,hall_2014_understanding,iqbal_2021_how, bragana_2024_anlise}.

However, monitoring natural disasters presents substantial technical challenges. The vast scales involved, coupled with harsh environmental conditions and the need for real-time data, demand robust, cost-effective, and scalable technological solutions. In this context, Wireless Sensor Networks (WSNs) have emerged as a promising tool, offering significant advantages in terms of deployment flexibility, rapid response capabilities, and affordability compared to traditional monitoring infrastructures. WSNs enable dense spatial sampling and continuous data collection, which are critical for disaster monitoring applications.\cite{chen_2013_natural, ferreira_2023_conception, pule_2017_wireless}

To further enhance WSN capabilities in remote and large-scale environments, Low Power Wide Area Networks (LPWAN) like LoRaWAN have gained prominence. LoRaWAN is specifically designed for transmitting small amounts of data over long distances with extremely low power consumption, enabling sensor nodes to operate autonomously for up to 10 years . This makes it particularly suitable for hydrological monitoring systems in regions with limited infrastructure \cite{pule_2017_wireless, chen_2013_natural,ferreira_2023_conception}. 

This Master's project aims to develop a WSN for river water level monitoring using LoRa technology, focusing on the performance comparison of low-cost LiDAR and ultrasonic sensors. The project will involve designing and implementing a microcontroller-based sensor node capable of interfacing with various sensors, including TF-Luna, TF-Nova, JSN-SR04T, and HC-SR04.

\section{Objectives}

The primary objective of this Master's project is to design, implement, and rigorously evaluate a LoRa-based WSN for river water level monitoring, focusing on a comprehensive performance comparison of selected low-cost LiDAR and ultrasonic sensors.

\section{Specific Objectives}

\begin{enumerate}
    \item To develop a microcontroller-based sensor node capable of interfacing with TF-Luna, TF-Nova, JSN-SR04T, and HC-SR04 sensors;
    \item To implement robust firmware for synchronized data acquisition, local data processing, and efficient LoRa/LoRaWAN data transmission.
    \item To conduct laboratory experiments to quantify and compare the accuracy, precision, range, resolution, and susceptibility to common interferences (e.g., temperature shifts, target surface variations, water conditions) of each sensor;
    \item To deploy the developed sensor nodes in a real environment, assessing their performance, data integrity, and operational resilience against ambient environmental conditions;
    \item To analyze the collected field and lab data to provide a assessment of the sensors, identifying their respective strengths, weaknesses, and optimal operational conditions for river stage monitoring;
    \item To offer evidence-based recommendations for sensor selection, node design, and LoRa network deployment strategies for scalable and cost-effective hydrological monitoring systems in similar regional contexts;
\end{enumerate}

\section{Structure of the Work}

This thesis is structured as follows:
\begin{itemize}
    \item \textbf{Chapter 2: Literature Review} - A comprehensive review of existing literature on WSNs, LoRa technology, and sensor technologies for hydrological monitoring.
    \item \textbf{Chapter 3: Materials and Methods} - Detailed description of the hardware and software design, sensor selection criteria, experimental setup, and data analysis methods.
    \item \textbf{Chapter 4: Laboratory Experiments} - Presentation and analysis of laboratory results comparing the selected sensors under controlled conditions.
    \item \textbf{Chapter 5: Field Deployment} - Description of the field deployment process, network setup, and performance evaluation in a real riverine environment.
    \item \textbf{Chapter 6: Discussion and Conclusion} - Summary of findings, implications for future research, and recommendations for practical applications in water level monitoring.
\end{itemize}



	
	% 2 - Desenvolvimento
	\chapter{Fundamentação Teórica} \label{cap:fundamentacao}

Esta seção é dedicada à exploração e discussão dos conceitos e teorias que embasam este trabalho. Neste capítulo, serão apresentados os principais temas e tecnologias envolvidos, como HTTP, NodeJS, \textit{websockets}, e APIs. A discussão será organizada de maneira progressiva, começando pelos conceitos mais gerais e avançando para os aspectos mais específicos e técnicos, como as bibliotecas e \textit{frameworks} utilizados.

\section{Automação de Serviços}

Automação de processos, se refere ao uso de tecnologia para realizar tarefas rotineiras e repetitivas de maneira eficiente e sem erros, e é um aspecto fundamental da automação de serviços.

A automação de serviços é uma tendência em crescimento nos últimos anos, especialmente na área de comércio eletrônico. Ela busca tornar processos mais eficientes, reduzir custos e oferecer melhores experiências aos clientes. Através da automatização, é possível reduzir erros e aumentar a precisão e rapidez no atendimento ao cliente. Com isso, os estabelecimentos podem oferecer um serviço de qualidade, com maior agilidade e menor tempo de espera.

A pandemia de COVID-19 acelerou a necessidade de digitalização em muitos setores. Com as restrições de movimento e o fechamento de lojas físicas, muitos negócios tiveram que se adaptar rapidamente para oferecer seus serviços online. Isso levou a um aumento na demanda por automação de serviços, à medida que as empresas procuravam maneiras de continuar operando de forma eficiente em um ambiente digital. De acordo com um estudo de Sousa, Silva e Araújo (2008), a automação não é apenas uma parte do processo; é um ecossistema em desenvolvimento em que a \glsxtrfull{IA} e a \gls{IoT} trabalham juntas para criar uma força de trabalho mais inteligente, eficiente e produtiva \cite{Gourley2002}.


\section{Comunicação HTTP e Web} \label{sec:http}

O protocolo \glsxtrfull{HTTP} é um dos principais métodos para transferência de dados via \textit{web}. Trata-se de um protocolo de cliente-servidor, o que significa que as solicitações são iniciadas pelo cliente, geralmente o navegador da \textit{web}, e um documento completo é reconstruído a partir dos diferentes recursos buscados, como texto, descrição de \textit{layout}, imagens, vídeos, \textit{scripts} e mais \cite{mdn_http_overview}, a Figura \ref{fig:http-get} ilustra esse processo.

\begin{figure}[h]
\centering
\caption{Buscando dados da Web.}
\includegraphics[width=0.8\textwidth]{figuras/fetching_a_page.png}
\fonte{\cite{mdn_http_overview}.}
\label{fig:http-get}
\end{figure}

O \gls{HTTP} é um protocolo sem estado, o que significa que cada requisição é independente das outras. Isso permite que a web seja altamente escalável, pois os servidores não precisam manter informações sobre cada usuário.

A \gls{WWW} foi criada em 1989 por Tim Berners-Lee, um cientista da computação britânico que trabalhava no CERN, o laboratório de física de partículas na Suíça. O objetivo era criar uma maneira fácil de compartilhar informações entre os cientistas que trabalhavam em diferentes universidades e institutos ao redor do mundo. O \gls{HTML} foi a linguagem de marcação que Berners-Lee desenvolveu para criar páginas \textit{web}. A Figura \ref{fig:http-html} demonstra a conexão entre estas tecnologias e como elas se encaixam para produzir a \textit{web}.

\begin{figure}[h]
\centering
\caption{Camadas da \textit{web}.}
\includegraphics[width=0.8\textwidth]{figuras/http-layers.png}
\fonte{\cite{mdn_http_overview}.}
\label{fig:http-html}
\end{figure}

Com o tempo, o \gls{HTTP} evoluiu e novas versões foram lançadas. A versão mais recente é o \gls{HTTP}/3, que oferece melhor desempenho e segurança em comparação com as versões anteriores. No entanto, o \gls{HTTP}/1.1 e \gls{HTTP}/2 ainda são amplamente utilizados na \textit{web} \cite{HttpIEEE}.

Um conceito importante para a comunicação HTTP são os métodos, os principais métodos HTTP são:

\begin{itemize}
\item \textbf{GET:} Solicita um recurso do servidor. Este é o método mais comum e é usado para solicitar a visualização de páginas da web.
\item \textbf{POST:} Envia dados para o servidor para criar um novo recurso. Os dados são incluídos no corpo da solicitação.
\item \textbf{PUT:} Envia dados para o servidor para atualizar um recurso existente. Os dados são incluídos no corpo da solicitação.
\item \textbf{DELETE:} Solicita a exclusão de um recurso no servidor.
\item \textbf{OPTIONS:} Solicita informações sobre os métodos de comunicação disponíveis para um recurso ou para o servidor em geral.
\item \textbf{PATCH:} Aplica modificações parciais a um recurso.
\item \textbf{CONNECT:} É usado para abrir uma conexão de rede bidirecional com o recurso solicitado. Geralmente é usado para acesso SSL (HTTPS).
\end{itemize}

Esses métodos são definidos no protocolo HTTP e são usados para indicar a ação desejada a ser realizada no recurso especificado. Eles formam a base da interação entre o cliente e o servidor na web \cite{HttpIEEE}.

\section{API}

Uma \glsxtrfull{API} é um conjunto de rotinas, protocolos e ferramentas para construção de software e aplicações. Ela define como os componentes de software devem interagir entre si, permitindo que diferentes aplicações possam se comunicar e compartilhar informações \cite{Mulloy2012}.

\subsection{REST API}

\glsxtrfull{REST} é um estilo arquitetural para sistemas distribuídos, baseado no protocolo HTTP. Ele define um conjunto de restrições que devem ser seguidas para que as aplicações possam se comunicar de forma eficiente e escalável. Uma das principais características do REST é a separação entre cliente e servidor, onde o cliente faz requisições ao servidor para acessar recursos, e o servidor responde com uma representação do estado atual do recurso solicitado. A Figura \ref{fig:rest} mostra a arquitetura de uma REST API.

\begin{figure}[h]
\centering
\caption{API Rest.}
\includegraphics[width=0.8\textwidth]{figuras/rest-api.png}
\fonte{\cite{restAPI}.}
\label{fig:rest}
\end{figure}

O \gls{REST} é um elemento fundamental na construção de APIs modernas, devido à sua simplicidade e eficiência. Ele permite que os desenvolvedores criem APIs que podem ser facilmente consumidas por diferentes clientes, incluindo navegadores web, aplicativos móveis e outros servidores. Além disso, o \gls{REST} é independente de linguagem, o que significa que pode ser usado com qualquer linguagem de programação que suporte \gls{HTTP}.

O \gls{REST} foi proposto por Roy Fielding em sua tese de doutorado em 2000. Fielding é um dos principais contribuidores para o desenvolvimento do protocolo HTTP e co-fundador da Apache \gls{HTTP} Server Project. Em sua tese, Fielding descreveu o REST como um conjunto de princípios arquiteturais que podem ser usados para projetar sistemas distribuídos que são escaláveis, eficientes e fáceis de modificar e manter.

Os métodos HTTP citados na Seção \ref{sec:http} são incorporados ao REST. Eles formam um estilo arquitetural que utiliza métodos com a mesma semântica para permitir a construção de APIS, com rotas do tipo POST, GET, DELETE etc.

Desde então, o \gls{REST} se tornou o estilo arquitetural mais popular para a construção de APIs na web. Ele é usado por muitas grandes empresas, incluindo Google, Facebook e Twitter, para fornecer acesso programático aos seus serviços \cite{Richardson2007}.

\subsection{APIs de Pagamento}

As \gls{API}s de pagamento surgiram como uma necessidade para facilitar as transações online. No início, as APIs de pagamento eram principalmente usadas para processar pagamentos com cartão de crédito. Empresas como a PayPal foram pioneiras nesse campo, fornecendo APIs que permitiam aos comerciantes aceitar pagamentos com cartão de crédito em seus sites.

Com o tempo, as APIs de pagamento evoluíram para suportar uma variedade de métodos de pagamento. Isso inclui não apenas cartões de crédito, mas também débito direto, pagamentos móveis e até mesmo cripto-moedas. Além disso, as APIs de pagamento também começaram a oferecer funcionalidades adicionais, como suporte para pagamentos recorrentes, reembolsos, e a capacidade de gerenciar várias moedas.

No Brasil, intermediários de pagamento como o Mercado Pago e o PagSeguro oferecem APIs que permitem aos comerciantes aceitar uma variedade de métodos de pagamento, incluindo boleto bancário e transferências bancárias. Recentemente, com a introdução do PIX, um sistema de pagamentos instantâneos operado pelo Banco Central do Brasil, esses intermediários também começaram a oferecer APIs que suportam PIX, permitindo transações quase instantâneas \cite{Aue2018, Adyen}.

\section{Sockets e WebSockets}
Os \textit{sockets e webSockets} são tecnologias fundamentais para a comunicação em tempo real na internet. \textit{Sockets} são um mecanismo de comunicação bidirecional entre dois nós em uma rede, permitindo a troca de dados em tempo real. Eles são amplamente utilizados em sistemas distribuídos e aplicações de rede para estabelecer conexões entre servidores e clientes \cite{Attoui2000}.

Os \textit{websockets}, por outro lado, são uma extensão dos \textit{sockets}, projetados especificamente para comunicação em tempo real na web. Eles superam as limitações das soluções tradicionais de comunicação em tempo real na web, como \textit{polling e long-polling}, fornecendo um mecanismo eficaz para comunicação bidirecional sustentada entre o cliente e o servidor. A tecnologia \textit{websockets} permite uma comunicação mais eficiente, reduzindo o tráfego de rede e a latência, tornando-a ideal para aplicações que exigem interações em tempo real \cite{Liu2012}.

\section{Tokens e JWT}

A autenticação e a autorização são componentes críticos de qualquer aplicação segura. \textit{Tokens}, particularmente \glsxtrfull{JWT}, são usados para transmitir informações de forma segura entre partes. \gls{JWT} é um padrão aberto (RFC 7519) que define uma maneira compacta e independente de transmitir informações entre partes como um \gls{JSON}. Essas informações podem ser verificadas e confiáveis porque são assinadas digitalmente. JWT pode ser assinado usando um segredo (com o algoritmo HMAC) ou um par de chaves pública/privada usando RSA ou ECDSA \cite{JWT2023}.

\section{QR Code}

O código \glsxtrfull{QR} é um código de barras bidimensional que pode armazenar informações em um formato legível por máquina. Ele foi desenvolvido para permitir a leitura rápida e eficiente de informações por dispositivos eletrônicos, como \textit{smartphones e tablets}.

Amplamente utilizado em várias aplicações, o QR Code se tornou muito comum no rastreamento de produtos, gerenciamento de inventário e marketing \cite{QRCode2023}. A Figura \ref{fig:qr} demonstra um QR que foi gerado com um \textit{link}.

\begin{figure}[h]
\centering
\caption{QR Code com link para loja.}
\includegraphics[width=0.4\textwidth]{figuras/qr.png}
\fonte{Criado pelo autor.}
\label{fig:qr}
\end{figure}

Esta codificação é gerada através de um processo que transforma a informação desejada (como um \textit{link} para um site) em um padrão de pontos pretos e brancos. Este padrão é lido por um \textit{scanner} (geralmente uma câmera de \textit{smartphones} com um aplicativo de leitura de QR Code), que decodifica a informação e realiza a ação correspondente (como abrir um \textit{link} em um navegador).

\section{Internet das Coisas (IoT)}
\gls{IoT} é discutido como uma tecnologia emergente que transforma objetos do mundo real em objetos virtuais inteligentes. \gls{IoT} visa unificar tudo em nosso mundo sob uma infraestrutura comum, não apenas nos dando controle sobre as coisas ao nosso redor, mas também nos mantendo informados sobre o estado dessas coisas \cite{Madakam:2015}. 

Isso torna possível a criação de dispositivos conectados a internet, capazes de monitorar e controlar processos e equipamentos remotamente. \gls{IoT} tem um papel crucial na transformação digital e na criação de cidades inteligentes, permitindo a coleta de dados em tempo real e a tomada de decisões baseada em dados. A Figura \ref{fig:iot} representa como diversos dispositivos se conectam a internet através da IoT.

\begin{figure}[h]
\centering
\caption{Internet of Things.}
\includegraphics[width=0.8\textwidth]{figuras/iot.png}
\fonte{\cite{iot123}.}
\label{fig:iot}
\end{figure}

\section{UUID}

Um \glsxtrfull{UUID} é um identificador único de 128 bits que é usado para identificar informações de forma única em um sistema de computação distribuído \cite{Leach2005}. Ele é gerado de forma aleatória, tornando-o altamente improvável de ser duplicado. O formato de UUID mais comum é o \gls{UUID} versão 4, que utiliza a geração aleatória de números para criar uma identificação única. O UUID é amplamente utilizado em sistemas distribuídos, como bancos de dados, sistemas de mensagens e sistemas de arquivos distribuídos \cite{Leach2005}.

Essa ferramenta permite que cada elemento seja rastreado e gerenciado de forma única, evitando conflitos ou duplicatas no sistema. Além disso, aumenta a segurança, visto que dificulta tentativas, por parte de usuários mal intencionados, de manipular algum elemento do sistema.

\section{Tecnologias da Aplicação Web}

Neste trabalho, serão utilizadas diversas tecnologias e bibliotecas. As mais importantes são explicadas nas seções a seguir. 

\subsection{TypeScript}

TypeScript é uma linguagem de programação que estende o JavaScript, adicionando tipos estáticos. Sua principal motivação é permitir o desenvolvimento de aplicações JavaScript em larga escala de maneira mais eficiente. O TypeScript introduz um sistema de módulos, classes e interfaces, além de um sistema de tipos gradual e robusto. Essas características permitem que os desenvolvedores escrevam código JavaScript de maneira mais clara e estruturada, facilitando a compreensão e a manutenção do código \cite[p.~257]{Bierman:2014}.

O TypeScript oferece ferramentas que auxiliam na construção do código, como a capacidade de listar os campos presentes em um objeto ou todos os métodos de uma classe. Isso facilita a navegação e a manipulação do código, especialmente em projetos de grande escala. Além disso, o TypeScript, por meio de seu sistema de tipos estáticos, é capaz de fornecer garantias sobre o comportamento do código, assegurando que os tipos de dados sejam consistentes ao longo do código, o que pode resultar em software mais confiável e de maior qualidade. Essas características tornam o TypeScript uma escolha popular para o desenvolvimento de aplicações web de grande escala, tanto para front-end quanto para back-end.

\subsection{SQL e Prisma}

\glsxtrfull{SQL} é uma linguagem de programação utilizada para gerenciar e manipular bancos de dados. Ela permite aos usuários criar, ler, atualizar e deletar dados em um banco de dados. SQL é uma linguagem padrão para bancos de dados relacionais e é usada em muitos sistemas de gerenciamento de bancos de dados, como MySQL, Oracle, PostgreSQL e SQL Server. SQL é uma linguagem essencial para desenvolvedores de software, analistas de dados e administradores de banco de dados, pois permite a interação eficiente com os dados armazenados em um banco de dados~\cite{SQL2023}.

No entanto, trabalhar diretamente com SQL pode ser complexo e propenso a erros. Para resolver isso, os desenvolvedores usam ferramentas chamadas mapeadores objeto-relacional (ORMs). Um ORM permite que os desenvolvedores interajam com o banco de dados usando o paradigma de programação orientada a objetos, o que é mais intuitivo e seguro para muitos desenvolvedores.

O Prisma é um ORM moderno e poderoso que permite o acesso a bancos de dados através de uma interface simples e intuitiva. Ele oferece diversas funcionalidades, como migrações de banco de dados, controle de versão de esquema e consultas otimizadas, permitindo um acesso rápido e eficiente aos dados. O Prisma pode ser usado para acessar bancos de dados de forma eficiente e segura, com suporte para várias funcionalidades, como migrações automáticas, cliente de banco de dados com tipagem segura e navegador de banco de dados visual \cite{Prisma2023}.

\subsection{Zod}

Zod é uma biblioteca para TypeScript que ajuda a garantir que os dados em uma aplicação estejam corretos e seguros. Ele faz isso através do uso de "esquemas", que são como modelos que descrevem como os dados devem ser estruturados, incluindo o tipo de dados e regras. Por exemplo, um esquema pode especificar que um item em uma loja online deve ter um nome (caracteres), uma descrição (caracteres) e um preço (número). O trecho de Código \ref{code:zod} mostra como esse elemento pode ser representado através dessa biblioteca, criando um tipo de dado chamado ``itemSchema''.

\renewcommand{\lstlistingname}{Código Fonte}

\begin{figure}[h]
\begin{lstlisting}[caption={Exemplo de uso da biblioteca Zod.},label={code:zod}]
import z from 'zod';

export const itemSchema = z.object({
  id: z.string().uuid(),
  name: z.string(),
  description: z.string().min(4),
  price: z.number(),
});
\end{lstlisting}
\fonte{Criado pelo autor.}
\end{figure}

Com Zod, é possível definir esses esquemas e usar a biblioteca para verificar se os dados de entrada e saída da aplicação correspondem a esses esquemas. Isso é especialmente útil em aplicações web, onde os dados de entrada geralmente vêm de usuários e podem ser imprevisíveis.

Ao usar Zod para validar esses dados, é possível prevenir muitos erros e vulnerabilidades de segurança, tornando a aplicação mais robusta e confiável \cite{Zod2023}.

\subsection{tRPC}
O \glsxtrfull{tRPC} é uma biblioteca leve que permite a criação de APIs totalmente seguras em termos de tipo \cite{tRPC}. Ele se apresenta como uma alternativa a outras tecnologias de chamada de procedimento remoto, como REST, GraphQL e gRPC.

Ele permite que os desenvolvedores criem APIs, sem a necessidade de manter manualmente as definições do formato dos dados entre o cliente e o servidor. Isso é conseguido através da geração automática de definições de tipo com base no código do servidor. Em outras palavras, o tRPC entende quais dados devem ser enviados na requisição, com base no código do servidor, eliminando a necessidade de manutenção manual dessas definições \cite{tRPC}. Complementando o que foi exibido anteriormente com o Zod, o Código \ref{code:tRPC} demonstra a implementação de uma rota backend que utiliza tRPC e Zod.

\begin{figure}[h]
\begin{lstlisting}[caption={Exemplo de uso da biblioteca tRPC no backend.},label={code:tRPC}]

export const itemsRouter = createTRPCRouter({
  create: publicProcedure
    .input(
      z.object({
        name: z.string(),
        description: z.string().min(4),
        price: z.number(),
      })
    )
    .mutation(({ ctx, input }) => {
      return ctx.prisma.items.create({
        data: {
          description: input.description,
          price: input.price,
          name: input.name,
        },
      });
    }),
});
\end{lstlisting}
\fonte{Criado pelo autor.}
\end{figure}

O Código \ref{code:trpc2}, corresponde a uma implementação do tRPC no frontend, que utiliza esta rota para enviar informações.

\begin{figure}[h]
\begin{lstlisting}[caption={Exemplo de uso da biblioteca tRPC no frontend.}, label={code:trpc2}]
export const itemsRouter = createTRPCRouter({
   // importa a rota da api trpc
  const createItem = api.items.create.useMutation({
    onSuccess: (createdItem) => {
      toast.success("Item criado com sucesso", {
        position: "top-right",
        autoClose: 3000,
        theme: "colored",
      });
      },
      onError: (err) => {
      toast.error("Erro ao criar item", {
        position: "top-right",
        autoClose: 5000,
        theme: "colored",
      });
      }
    });

    // utilizacao dela para enviar a requisicao
    createItem.mutate({
        name: values.name,
        description: values.description,
        price: values.price,
      });
\end{lstlisting}
\fonte{Criado pelo autor.}
\end{figure}

No contexto de desenvolvimento full stack, o tRPC se destaca como uma biblioteca fundamental para a construção tanto do servidor quanto do cliente. De acordo com o trabalho de \cite{Nivasalo2022}, o tRPC se mostrou uma excelente biblioteca para se basear, pois reduziu significativamente o tempo de desenvolvimento ao tornar extremamente fácil a implementação de chamadas de endpoint da API para o frontend.

Por fim, vale ressaltar que o tRPC é um projeto de código aberto e gratuito para uso, o que o torna uma excelente opção para desenvolvedores e empresas que buscam uma solução eficiente e econômica para a criação de APIs seguras em termos de tipo.

\subsection{Fastify}

Fastify é um \textit{framework} web eficiente para Node.js, projetado para ser o mais rápido possível, tanto em termos de tempo de execução quanto de velocidade de desenvolvimento \cite{fastify}. Ele fornece um conjunto robusto de recursos para construir aplicações web e é totalmente extensível com seu sistema de \textit{plugins}. Fastify também oferece um modelo de roteamento fácil de usar e suporte para manipulação de solicitações e respostas HTTP, tornando-o uma escolha popular para muitos desenvolvedores de Node.js.

\subsection{Next.js}

Next.js é um \textit{framework} JavaScript para sistemas baseados em React, que permite a criação de páginas estáticas e dinâmicas, bem como a geração de conteúdo sob demanda, proporcionando uma experiência de carregamento mais rápida e eficiente para o usuário. Este \textit{framework} pode ser usado para criar aplicações web eficientes e escaláveis, com suporte para várias funcionalidades, como roteamento por arquivo, \textit{streaming} de HTML dinâmico e suporte a CSS \cite{Nextjs2023}.

Amplamente utilizado e reconhecido na indústria de desenvolvimento web, o NextJs é o 14º maior projeto no GitHub e é considerado o \textit{framework} ReactJS número 1, com mais de 100.000 estrelas no GitHub. Grandes empresas como Notion e Twitch utilizam o Next.js em suas aplicações, o que demonstra a confiabilidade e a maturidade do \textit{framework} \cite{Vercel2023}.

Além disso, o Next.js é recomendado na documentação oficial do React para projetos  que se beneficiariam de suas características específicas, como renderização do lado do servidor e otimizações de desempenho. Isso indica a importância e a relevância do Next.js no ecossistema de desenvolvimento web atual. A adoção do Next.js neste trabalho é uma escolha estratégica que visa aproveitar essas vantagens e recursos poderosos para criar uma aplicação web robusta e eficiente.

\section{NodeMCU ESP8266}

O NodeMCU ESP8266 é um microcontrolador baseado na plataforma ESP8266, que possui conectividade Wi-Fi integrada. Ele é bastante utilizado em projetos de \gls{IoT} devido à sua facilidade de programação, baixo custo e recursos avançados. O ESP8266, demonstrado na Figura \ref{fig:node}, é especialmente adequado para tais aplicações devido à sua capacidade de operar em baixa potência, o que é crucial para dispositivos alimentados por bateria, e sua capacidade de se conectar a redes Wi-Fi, permitindo a comunicação com a internet e outros dispositivos na rede  \cite{Kolban2016}.

\begin{figure}[h]
\centering
\caption{ESP8266 NodeMcu.}
\includegraphics[width=0.5\textwidth]{figuras/nodemcu.jpg}
\fonte{\cite{nodemcu}.}
\label{fig:node}
\end{figure}

\subsection{C++ e Bibliotecas}

O microcontrolador NodeMCU, pode ser programado em C++ e as bibliotecas relevantes para este projeto estão listadas a seguir.
\begin{itemize}
\item \textbf{ESP8266WiFi.h:} Esta biblioteca permite que o NodeMCU se conecte a redes Wi-Fi. Ela fornece funções para conectar, desconectar e verificar o status da conexão WiFi. No código fornecido, essa biblioteca é usada para conectar o NodeMCU à rede Wi-Fi especificada pelas constantes ``ssid'' e ``password''.

\item \textbf{WebSocketsClient.h:} Esta biblioteca permite que o NodeMCU se comunique com servidores \textit{websocket}. \textit{Websocket} é um protocolo que permite comunicação bidirecional em tempo real entre clientes e servidores. No código fornecido, essa biblioteca é usada para conectar o NodeMCU a um servidor \textit{websocket} e enviar/receber mensagens.
\end{itemize}


% ----------------------------------------------------------




	
	% 3 - Seção
	% ----------------------------------------------------------
\chapter{Arquitetura do sistema}\label{cap:arquitetura}
% ----------------------------------------------------------
 
Este capítulo delineia a arquitetura do sistema proposto, fornecendo justificativas para as decisões arquitetônicas tomadas. A proposta em questão visa desenvolver uma alternativa inovadora para a automação de pagamentos, incorporando tecnologias contemporâneas e princípios da Web 3.0, como \textit{tokens} e criptografia de dados. Elementos como a computação em nuvem, a Internet das Coisas (\gls{IoT}) e o uso de \textit{tokens} únicos, como JWT e UUID, são integrados com o objetivo de estabelecer uma infraestrutura eficaz e segura.

O sistema é composto por três componentes principais: uma aplicação \textit{web fullstack}, um servidor de \textit{sockets} e um agente de distribuição. Esses componentes interagem entre si para fornecer a funcionalidade desejada. A Figura \ref{fig:system_architecture} ilustra a arquitetura do sistema.

\begin{figure}[h]
\centering
\caption{Arquitetura do Sistema.}
\includegraphics[width=1\textwidth]{figuras/Arquitetura.png}
\fonte{Criado pelo autor.}
\label{fig:system_architecture}
\end{figure}

Cada componente tem um papel específico no sistema e trabalha em conjunto com os outros para fornecer a funcionalidade desejada. A arquitetura foi projetada para ser escalável e eficiente, e alcança através de múltiplos meios. A implementação de \textit{sockets} em um servidor separado, evita a necessidade de \textit{pooling}, reduzindo o número de requisições ao servidor e a separação de componentes, como o banco de dados e servidor de \textit{sockets}, permite que sejam escalados separadamente para atender as demandas.

As seções 3.1 a 3.5 detalham cada um dos elementos da arquitetura. 

\section{Aplicação Web FullStack}
Este elemento é o centro do sistema e coordena todos os demais. Está dividido entre \textit{frontend} e \textit{backend} para melhor representar suas funcionalidades, porém é representado de forma única pois é dessa forma que se dá sua implementação e desenvolvimento.

\begin{enumerate}

\item \textbf{Frontend:} Esta é a interface do sistema, que é executada no dispositivo do usuário, onde o mesmo pode interagir com as lojas online cadastradas no sistema. Além disso, permite a criação, edição, visualização e exclusão de contas de administradores, lojas, itens e pedidos. É através desta interface que todas as interações do usuário com o sistema ocorrem. Estas ações serão efetuadas através de requisições para o \textit{backend} que é executado no servidor.
\item \textbf{Backend:} Esta é a \gls{API} que processa as solicitações vindas do \textit{frontend}. É neste elemento que a conexão com o banco de dados é estabelecida para recuperar e armazenar informações. Além disso, o \textit{backend} é responsável por se comunicar com a API do Mercado Pago para registrar pedidos e receber os links que permitem aos usuários efetuar pagamentos. Este componente também se conecta com a API do sistema responsável pelo gerenciamento dos \textit{websockets}, garantindo que as notificações sejam enviadas sempre que o status de um pedido for atualizado.
\end{enumerate}

\section{Servidor de Sockets}

A decisão de utilizar \textit{sockets} foi tomada para permitir que os Agentes de Distribuição recebam informações de maneira contínua. A adoção dessa tecnologia resulta em uma redução significativa no número de requisições ao servidor, especialmente quando comparada à comunicação via múltiplas requisições HTTP feitas por diversos dispositivos em intervalos curtos de tempo.

A separação deste componente da aplicação \textit{fullstack} é necessária devido às características das conexões \textit{socket}. Este tipo de conexão mantém uma comunicação contínua, o que gera uma complexidade maior tanto no desenvolvimento quanto na implantação, se comparada a uma API do tipo REST. Isso ocorre pois, para implementar \textit{sockets}, o servidor deve manter as conexões ativas indeterminadamente, algo que não é garantido por todos os provedores de hospedagem em nuvem e também não é indicado para uso na API da aplicação \textit{fullstack}, com NextJS. Este tipo de API não foi projetada para manter conexões ativas por longos períodos de tempo, e por mais que possam estabelecer conexões \textit{sockets}, isto não é uma prática recomendada em sua documentação e pelos criadores. Por esses motivos, optou-se por criar este elemento separadamente.

Este componente da arquitetura é um servidor independente, que hospeda uma API capaz de receber comunicações HTTP e  \textit{websocket}. Esta \gls{API} estabelece conexões \textit{socket} com os microcontroladores e as mantém ativas enquanto houver algum dispositivo conectado. Para garantir a segurança, as conexões devem ser autenticadas com uma chave \gls{UUID} incluída no cabeçalho de cada requisição. Além disso, cada \textit{socket} possui um ID único, permitindo a distinção entre diferentes pontos de venda.

\section{Agente de Distribuição}

Os agentes de distribuição são dispositivos que estabelecem conexão com o servidor de \textit{sockets} para receber informações e responder aos pedidos. No contexto deste trabalho, a ênfase será dada à conexão e ao recebimento de dados provenientes da API.

A implementação destes agentes de distribuição, capazes de entregar um produto ou realizar alguma tarefa paga através do sistema pode variar significativamente e não será o foco deste trabalho. Estes dispositivos incluem mas não se limitam á: aplicativos, microcontroladores, dispensers, entregadores ou uma tela que apresente os itens pedidos a uma equipe de funcionários. Portanto, a principal preocupação aqui é garantir uma comunicação eficaz e segura entre um agente de distribuição e o servidor de sockets.

\section{Mercado Pago}
Esta é a aplicação externa utilizada para efetivar os pagamentos, tem \textit{frontend e backend}. O \textit{frontend} permite ao usuário fornecer os dados de pagamento e o \textit{backend} recebe os pedidos e envia notificações. Poderia ser substituída por outra API similar, porém seriam necessárias alterações significativas na estrutura do projeto, visto que APIs diferentes utilizam métodos de autenticação e estruturas de dados diferentes.

Quando um pedido é gerado no \textit{frontend} da aplicação \textit{web}, uma requisição é feita para a API do Mercado Pago contendo informações deste pedido e um token do sistema. Ao receber estas informações do pedido, a API retorna um link contendo o endereço de uma página onde o usuário pode realizar o pagamento. Assim que o pagamento é confirmado internamente pelo Mercado Pago, a API envia uma notificação para o \textit{backend} da aplicação web deste sistema, informando que houve uma atualização no pedido, e a aplicação web faz uma nova requisição para a API do Mercado Pago, para obter o status do pedido atualizado.

\section{Base de Dados}

É o banco de dados da aplicação, deve ser capaz de armazenar itens, pedidos, usuários, pontos de vendas e quaisquer outros elementos necessários ao sistema.

%---------------------------------------------------------------------------------------
\chapter{Implementação}\label{cap:desenvolvimento}
%---------------------------------------------------------------------------------------

Este capítulo descreve o desenvolvimento do projeto, que envolveu a aplicação dos conceitos apresentados na fase de fundamentação teórica e a aplicação da arquitetura definida. O capítulo é dividido em quatro seções: a Seção \ref{cap:bancodedados} fala sobre o banco de dados, a Seção \ref{cap:fullstack} apresenta a aplicação \textit{web full-stack}, a Seção \ref{cap:sockets} traz a definição o servidor de \textit{sockets} e a Seção \ref{cap:agent} aborda o agente de distribuição.

\section{Banco de Dados} \label{cap:bancodedados}

A primeira etapa do processo de desenvolvimento foi a modelagem do banco de dados relacional. O modelo definido é composto por quatro entidades fundamentais: usuários, pontos de venda (PDVs), pedidos e itens. Cada entidade possui seus atributos específicos e suas relações com as outras entidades. O \glsxtrfull{ERD} do banco de dados, obtido através da ferramenta DBeaver está representado pela Figura \ref{fig:database}. A seguir são descritas cada uma das entidades presentes na Figura \ref{fig:database}:

\begin{figure}
	\caption{\label{fig:database}ERD - Projeto.}
	\begin{center}
		\includegraphics{figuras/database.png}
	\end{center}
	\fonte{Criado pelo autor.}
\end{figure}

\begin{itemize}
    \item \textit{User} (Usuário): Este modelo representa um usuário do sistema. Os usuários têm uma identificação única (id), e-mail, senha, cpf/cnpj, nome, cargo (role), além dos campos de controle createdAt e updatedAt.
    \item RoleEnumType: Este é um tipo de enumeração que define os possíveis papéis que um usuário pode ter: user (usuário) ou admin (administrador).
    \item PDV (Ponto de Venda): Representa um ponto de venda dentro do sistema. Cada PDV tem uma identificação única, estado (ativo ou não), tipo, empresa (company), login, senha e duas listas de itens e pedidos relacionados a ele.
    \item \textit{Order} (Pedido): Representa um pedido feito no sistema. Cada pedido tem uma identificação única, uma lista de itens (ItemsOnOrder), preço, link de pagamento, identificação de pagamento, status, identificação do ponto de venda onde foi feito e os campos de controle createdAt e updatedAt.
    \item OrderStatusEnumType: Este é um tipo de enumeração que define os possíveis estados que um pedido pode ter: pending (pendente), approved (aprovado), accredited (creditado), delivered (entregue), canceled (cancelado).
    \item ItemsOnOrder (Itens no Pedido): Representa a relação entre pedidos e itens. Ele inclui a quantidade de cada item no pedido, o preço por item, a identificação do item, a identificação do pedido e a data em que o item foi atribuído ao pedido.
    \item ItemsOnPDV (Itens no Ponto de Venda): Representa a relação entre os pontos de venda e os itens. Ele inclui a quantidade de cada item no ponto de venda, a identificação do item, a identificação do ponto de venda e a data em que o item foi atribuído ao ponto de venda.
    \item Items (Itens): Representa um item que pode ser vendido em um ponto de venda e incluído em um pedido. Cada item tem uma identificação única, nome, descrição, preço e listas de pedidos e pontos de venda aos quais está associado.
    \item Prisma Migrations: Este modelo é utilizado pelo ORM Prisma para controlar as alterações na estrutura do banco, controlando quais migrations foram aplicadas no banco de dados, permitindo um versionamento deste.
\end{itemize}

Em termos de relacionamentos, os usuários (User) não estão diretamente associados a outras entidades. Os pedidos (Order) estão associados a um ponto de venda (PDV) numa relação muitos pra um, através da coluna pdvId na tabela Order, e de muitos para muitos com itens (Items), através da tabela ItemsOnOrder. Já os pontos de venda (PDV) tem uma associação de um para muitos com itens através da entidade ItemsOnPDV.

Optou-se pela utilização do PostgreSQL como banco de dados, e foram incluídas as entidades citadas anteriormente, seguindo a sintaxe do Prisma de forma a criar e se comunicar com o banco de dados. O Código \ref{code:prisma} exibe a declaração da tabela User, usando a sintaxe do Prisma.

\begin{figure}[h]
\begin{lstlisting}[caption={Exemplo de um schema no Prisma.}, label={code:prisma}]
generator client {
    provider = "prisma-client-js"
}

datasource db {
    provider = "postgresql"
    url      = env("DATABASE_URL")
}

model User {
    id        String        @id @unique @default(uuid())
    email     String        
    password  String
    cpf_cnpj  String        @unique
    name      String
    role      RoleEnumType? @default(user)
    createdAt DateTime      @default(now())
    updatedAt DateTime      @updatedAt
}

enum RoleEnumType {
    user
    admin
}
    
\end{lstlisting}

\fonte{Criado pelo autor.}
\end{figure}

\section{Aplicação Web Full Stack} \label{cap:fullstack}

A aplicação web foi desenvolvida utilizando o \textit{framework} Next.js, que permite a criação de aplicações fullstack com React. A IDE escolhida foi o Visual Studio Code (VS Code). Como o Next.js é um \textit{framework full-stack}, tanto o \textit{front-end} quanto o \textit{back-end} foram desenvolvidos dentro de um projeto NextJS. Pode-se destacar algumas bibliotecas como o tRPC para criação das rotas da API no \textit{backend} e acesso a elas no \textit{frontend}, material UI que é uma biblioteca de componentes React de código aberto, Zod para verificação de dados e Prisma como \gls{ORM} para gerenciar o banco de dados. 

A aplicação possui três pontos de entrada para os usuários: a página para compras, a página de gestão de um ponto de vendas, e a página de gestão do sistema. A página de compras é pública, já as páginas de gestão de um ponto de vendas e gestão do sistema contam com autenticação, onde o usuário deve ter um login e senha autorizados. Cada ponto de vendas tem um acesso, conforme estabelecido no momento em que foi cadastrado, já para o acesso de gestão do sistema, múltiplos acessos podem ser cadastrados. 

\subsection{Fluxo de Compras}

A Figura \ref{fig:compra} mostra o fluxograma de compra para um cliente. Ao acessar a página de compras, o sistema segue um layout tradicional, listando todos os produtos em uma lista e exibindo para o usuário o preço total, como exibido na Figura \ref{fig:loja}. Quando o usuário finaliza a compra, é redirecionado para o pagamento através do Mercado Pago. A API do Mercado Pago foi escolhida por permitir o pagamento via PIX e por disponibilizar uma interface muito completa e confiável para o usuário, sendo uma marca conhecida no Brasil e que traz diversas garantias e verificações para o processo de pagamento.

\begin{figure}
	\caption{\label{fig:compra}Fluxograma de compra.}
	\begin{center}
		\includegraphics[width=0.8\textwidth]{figuras/Diagrama compra.png}
	\end{center}
	\fonte{Criado pelo autor.}
\end{figure}

\begin{figure}
	\caption{\label{fig:loja}Página da loja.}
	\begin{center}
		\includegraphics[width=\textwidth]{figuras/loja.png}
	\end{center}
	\fonte{Criado pelo autor.}
\end{figure}

Assim que o pagamento é efetuado, uma notificação é enviada para a API da aplicação Web, e a aplicação realiza uma requisição ao servidor do Mercado Pago para obter confirmação do pagamento.

Ao ser confirmado o pagamento, a aplicação Web envia uma requisição HTTP Post para a API que controla os Websockets, de forma a postar uma mensagem na conexão do ponto de vendas em que ocorreu a compra.

\subsection{Fluxo de Administração do Sistema}

A administração do sistema, apenas pode ser feita por usuários autorizados através da página de \textit{login}. Um administrador pode visualizar, criar, editar e excluir pontos de vendas e usuários administradores. Além disso, contas de administrador têm acesso à página ``log de pagamentos'',  que exibe as últimas notificações recebidas da API do Mercado Pago. A Figura \ref{fig:admin} representa o diagrama de administração do sistema.

\begin{figure}
	\caption{\label{fig:admin}Fluxograma administrativo.}
	\begin{center}
		\includegraphics[width=0.8\textwidth]{figuras/Diagrama administrativo.png}
	\end{center}
	\fonte{Criado pelo autor.}
\end{figure}

\subsection{Fluxo de Administração de Um Ponto de Vendas}

O controle de um ponto de vendas apenas pode ser acessado através da página de \textit{login} para controle de loja. A Figura \ref{fig:pdv} mostra um fluxograma com as opções para controle de um ponto de vendas. Através desse acesso o usuário pode visualizar, criar, editar e excluir itens de um ponto de venda, gerenciar e visualizar pedidos, e também pode acessar a página da loja ligada ao ponto de vendas, para conferir como está a interface onde os clientes realizam os pedidos. 

\begin{figure}
	\caption{\label{fig:pdv}Fluxograma do controle de um ponto de vendas.}
	\begin{center}
		\includegraphics[width=0.8\textwidth]{figuras/Diagrama de controle de loja.png}
	\end{center}
	\fonte{Criado pelo autor.}
\end{figure}

\subsection{\textit{Frontend}}

O \textit{frontend} foi desenvolvido utilizando React, uma biblioteca JavaScript para construção de interfaces de usuário. A interface foi projetada para ser intuitiva e simples. Ela utiliza no \textit{layout} componentes da biblioteca Material UI, uma biblioteca de componentes que implementa o Material Design do Google, e componentes criados com HTML e CSS. Foram criadas páginas para a listagem, criação, edição e exclusão de usuários e pontos de venda. A Figura \ref{fig:pdvs} mostra a tela do painel do administrador que permite o controle dos pontos de vendas, e representa o layout que foi seguido ao longo de todo o frontend para listar dados.

\begin{figure}
	\caption{\label{fig:pdvs}Tela de controle de PDVs.}
	\begin{center}
		\includegraphics[width=\textwidth]{figuras/tela pontos de vendas.png}
	\end{center}
	\fonte{Criado pelo autor.}
\end{figure}

Além do \textit{layout} anterior, usado para listagem, outro \textit{layout} que aparece em vários locais do sistema destina-se a coleta de informações, representado na Figura \ref{fig:crud}, usado para adicionar e editar informações. 

\begin{figure}
	\caption{\label{fig:crud}Tela para adicionar Item.}
	\begin{center}
		\includegraphics[width=\textwidth]{figuras/adicionar.png}
	\end{center}
	\fonte{Criado pelo autor.}
\end{figure}

\subsection{\textit{Backend}} 

A API \textit{backend} dentro da aplicação \textit{full-stack} foi dividida em grupos de rotas, acordo com as entidades do banco de dados envolvidas. Existem grupos de rotas para autenticação, itens, pedidos, pontos de vendas e usuários. Um exemplo de rota pode ser visto no Código \ref{code:prisma}. Trata-se da rota de \textit{login} que exige como entrada um objeto com os campos  ``email'' e  ``password''. Se esses parâmetros forem fornecidos corretamente, o processo descrito como \textit{mutation} é iniciado. Uma \textit{mutation} se refere a qualquer procedimento que altera o estado dos dados. Nesse caso, a ação desejada é realizada e a senha criptografada no banco é comparada com a senha recebida na rota. O Código \ref{code:login} mostra como a rota de login foi implementada e parte do processo de autenticação.

\begin{figure}[h]
\begin{lstlisting}[caption={Exemplo de uma rota tRPC.}, label={code:login}]
export const authRouter = createTRPCRouter({
  login: publicProcedure
    .input(
      z.object({
        email: z.string(),
        password: z.string().min(4),
      })
    )
    .mutation(async ({ input, ctx }) => {
      const { res } = ctx;
      const { email, password } = input;

      const databaseUser = await ctx.prisma.user.findFirst({
        where: { email: email },
      });
      if (!databaseUser) {
        throw new TRPCError({
          code: "NOT_FOUND",
          message: "Usuario nao encontrado",
        });
      }

      const isPasswordValid = bcrypt.compareSync(
        password,
        databaseUser.password
      );
\end{lstlisting}
\fonte{Criado pelo autor.}
\end{figure}

O tRPC serve como um gerenciador destas rotas. Diferente de uma API mais comum, onde o desenvolvedor especifica o caminho de cada rota, e precisa criar verificações para os tipos e respostas, o tRPC, em conjunto com o Zod, fornecem esta estrutura, cabendo ao desenvolver apenas definir quais dados devem ser recebidos, sem a necessidade de criar os meios de verificação. 

As rotas implementadas, além de serem utilizadas pelas páginas do frontend para enviar e receber dados, também fazem as conexões externas, como confirmação de pagamentos com a API do Mercado Pago e envio de mensagens para a API de Websockets.

O desenvolvimento da aplicação web resultou em um sistema funcional que atende aos requisitos estabelecidos na seção de objetivos deste trabalho. Foram criadas telas para o cadastro de usuários, pontos de venda e pedidos, além de telas para a visualização e edição dessas informações. O sistema conta com autenticação de usuários e utilização de \textit{tokens} JWT para garantir a segurança das informações.

A interface do sistema foi desenvolvida seguindo boas práticas de usabilidade e design, resultando em uma interface intuitiva e fácil de usar para os usuários finais.

\section{Servidor de \textit{Sockets}} \label{cap:sockets}
Este elemento do sistema é um servidor \textit{websocket} construído com a biblioteca Fastify para Node.js, com autenticação por chave de acesso e suporte a conexões \glsxtrfull{CORS}\footnote{CORS é um mecanismo que permite que muitos recursos (por exemplo, fontes, JavaScript, etc.) em uma página da web sejam solicitados de outro domínio fora do domínio da qual a origem da solicitação foi feita.}. Ele é encarregado da comunicação entre o servidor e os agentes de distribuição, permitindo que as atualizações sejam enviadas assim que ocorrem, sem a necessidade de solicitações constantes. Este método de comunicação foi escolhido por permitir uma melhor escalabilidade do sistema, reduzindo o número de requisições que precisam ser processadas pelo servidor.

Ao final do desenvolvimento, é possível criar conexões do tipo \textit{socket} privadas, e enviar mensagens para os \textit{sockets} via requisições HTTP, atendendo as necessidades de comunicação do sistema entre a API e os agentes de distribuição. A seguir será detalhado o funcionamento do servidor.

\subsection{Configurações iniciais e Criação do servidor \textit{websocket}}  As variáveis de ambiente são carregadas utilizando a biblioteca \texttt{dotenv} e instância do Fastify é criada, e configurada para permitir requisições de qualquer origem e para os métodos GET e POST.

Uma instância do servidor \textit{websocket} é criada usando a biblioteca \texttt{ws}. O servidor \textit{websocket} é configurado para não criar um servidor HTTP próprio, pois o servidor HTTP será fornecido pelo Fastify.

\subsection{Gerenciamento e Atualização de conexões \textit{websocket}} Uma lista para armazenar as conexões \textit{websocket} é criada. Cada conexão é identificada por uma \textit{string}, que é obtida do URL da requisição que abriu a conexão. O servidor \textit{websocket} é configurado para lidar com eventos de conexão, mensagem e fechamento. Quando uma nova conexão é estabelecida, o servidor registra a conexão em uma lista de conexões. Quando uma mensagem é recebida, o servidor a envia para a conexão a qual se destina. Quando uma conexão é fechada, o servidor remove a conexão do mapa.

O servidor Fastify é configurado para lidar com eventos de upgrade de conexões HTTP para \textit{websocket}. Quando uma requisição de upgrade é recebida, o servidor verifica se a chave de acesso fornecida nos cabeçalhos da requisição é válida. Se a chave de acesso for válida, a requisição de upgrade é passada para o servidor WebSocket. Caso contrário, a conexão é encerrada.

\subsection{Roteamento HTTP e Socket} 

Duas rotas HTTP são definidas:

\begin{itemize}
    \item \textbf{GET "/status"} - retorna uma mensagem indicando que o servidor está online.
    \item \textbf{POST "/post"} - é usada para enviar mensagens através das conexões WebSocket. Antes de tratar a requisição, a rota verifica se a chave de acesso fornecida nos cabeçalhos da requisição é válida. Se a chave de acesso for válida, a requisição é processada. A requisição deve incluir no corpo um ID de conexão e uma mensagem. A mensagem é enviada através da conexão socket correspondente ao ID fornecido.
\end{itemize}

Para conexão via WebSocket:
\begin{itemize}
    \item \textbf{ws://\{endereço do servidor\}/\{ID do socket\}} - conecta no socket com o ID passado como parâmetro. Similar a rota HTTP /post, antes de tratar a conexão, a rota verifica se a chave de acesso fornecida nos cabeçalhos da requisição é válida.
\end{itemize}

\section{Agente de distribuição} \label{cap:agent}

O desenvolvimento de um agente de distribuição se deu através de um microcontrolador NodeMCU ESP8266 e da plataforma Arduino IDE. O código do controlador desenvolvido em C++, inclui as bibliotecas ESP8266WiFi.h, que permite que o dispositivo se conecte com qualquer rede wifi, e WebSocketsClient.h que permite a conexão com \textit{websocket}. O microcontrolador se conecta no \textit{websocket}, com o UUID do ponto de vendas, e verifica cada mensagem recebida. Neste projeto a ênfase será na conexão e recebimento de dados do servidos de \textit{socket}, e o microcontrolador acende um led para demonstrar que recebeu a notificação. A Figura \ref{fig:diagrama conexao} apresenta o funcionamento do microcontrolador.

\begin{figure}
	\caption{\label{fig:diagrama conexao}Diagrama da conexão NodeMCU e servidor.}
	\begin{center}
		\includegraphics[width=\textwidth]{figuras/diagrama nodemcu.png}
	\end{center}
	\fonte{Criado pelo autor.}
\end{figure}

Para implementar um agente de distribuição, este deverá verificar pelos códigos de itens presentes no ponto de venda, e implementar o que for necessário para fazer a entrega de cada item.

O desenvolvimento do agente de distribuição no NodeMCU resultou em um dispositivo funcional capaz de se comunicar com a aplicação web e receber dados sobre o status dos pedidos. Foram realizados testes que provaram a integração do NodeMCU com a aplicação web e validaram o recebimento de dados sobre o status dos pedidos. Eles serão detalhados na Seção \ref{cap:host}.

Uma alternativa para este agente de distribuição automático, é a utilização da interface para controle de pedidos do ponto de vendas, exibida na Figura \ref{fig:pedidos}. Este método é chamado de ``manual'', e pode ser selecionado no momento de criação do ponto de vendas no sistema. Pontos de vendas manuais não precisam de um agente de distribuição e os pedidos devem ser atualizados manualmente. Eles não utilizam conexões no servidor de \textit{socket} e devem ser controlador inteiramente pelo usuário através da interface. O método apresentado anteriormente, com um agente de distribuição conectado via \textit{socket}, é definido como ``automated'' no momento da criação do ponto de vendas e é considerado o padrão.

\begin{figure}
    	\caption{\label{fig:pedidos}Tela de pedidos.}
    	\begin{center}
    		\includegraphics[width=0.8\textwidth]{figuras/pedidos.png}
    	\end{center}
    	\fonte{Criado pelo autor.}
\end{figure}

\chapter{Resultados}\label{cap:resultados}

Este capítulo apresenta os resultados alcançados com a execução e implementação do projeto. O capítulo é dividido em 3 seções: a Seção \ref{cap:test} trata dos testes de funcionamento, a Seção \ref{cap:host} trata da hospedagem e a Seção \ref{cap:final} trata das considerações finais.
    
\section{Resultados e testes da implantação em ambiente de produção} \label{cap:test}

Para testar o sistema e demonstrar na prática os resultados obtidos, será simulado o processo de compra de um café em um ponto de vendas programado para o teste. Os passos simulados são os seguintes:

\begin{itemize}
    \item Para fazer uma compra, o cliente deve acessar a página loja do ponto de venda em que deseja fazer a compra. A Figura \ref{fig:qrstore} mostra um QR code que redireciona para a loja de testes do sistema. O mesmo poderia estar presente em um ponto de vendas físico, ou o link\footnote{\url{https://quickpay.vercel.app/store/00f138a0-e8ac-4497-8eb9-057fcf2dec0d}} poderia ser fornecido ao cliente diretamente para vendas via chat virtual.


    \begin{figure}
    	\caption{\label{fig:qrstore}QR code para a loja de testes.}
    	\begin{center}
    		\includegraphics[width=0.4\textwidth]{figuras/qrcodestore.png}
    	\end{center}
    	\fonte{Criado pelo autor.}
    \end{figure}

    \item Na loja \textit{online}, o cliente seleciona o produto que deseja e clica em ``finalizar compra''. A Figura \ref{fig:testeloja} mostra a loja de testes. Ao clicar em finalizar compra uma mensagem é exibida, indicando ao usuário a página de pagamentos, Figura \ref{fig:avisopagamento}.

    \begin{figure}
    	\caption{\label{fig:testeloja}Loja de testes.}
    	\begin{center}
    		\includegraphics[width=0.8\textwidth]{figuras/testeloja.png}
    	\end{center}
    	\fonte{Criado pelo autor.}
    \end{figure}

    \begin{figure}
    	\caption{\label{fig:avisopagamento}Mensagem ao finalizar compra.}
    	\begin{center}
    		\includegraphics[width=0.8\textwidth]{figuras/avisopagamento.png}
    	\end{center}
    	\fonte{Criado pelo autor.}
    \end{figure}
    
    \item Caso o dispositivo do usuário permita, ele será redirecionado a página de pagamento. Caso contrário, ele pode pagar usando o link presente na mensagem. A página de pagamento é demonstrada na Figura \ref{fig:pagar}, nela o usuário pode selecionar o método de pagamento de sua preferência. O processo é garantido através da integração com o Mercado Pago.
    
    \begin{figure}
    	\caption{\label{fig:pagar}Tela de pagamento.}
    	\begin{center}
    		\includegraphics[width=0.8\textwidth]{figuras/pagamento.png}
    	\end{center}
    	\fonte{Criado pelo autor.}
    \end{figure}

    \item Quando o pagamento é recebido e confirmado pela API, o status do pedido é atualizado no sistema, ficando visível na interface. Caso seja um ponto de vendas do tipo automático, uma mensagem com o id do item adquirido é enviada no \textit{socket} do ponto de vendas.

    \item Caso o ponto de vendas tenha um agente de distribuição conectado, o pedido será entregue por este de forma automática. Caso o ponto de vendas seja do tipo manual, um responsável pelo ponto de vendas deve acompanhar os pedidos na interface do sistema, e realizar as entregas. Neste exemplo, o agente de distribuição é um NodeMCU que está programado para acender um led, de forma a demonstrar o recebimento do pedido, como exibido na Figura \ref{fig:nodeled}.

    \begin{figure}
    	\caption{\label{fig:nodeled}NodeMCU ao identificar ID esperado no socket.}
    	\begin{center}
    		\includegraphics[width=0.5\textwidth]{figuras/nodemcunotification.png}
    	\end{center}
    	\fonte{Criado pelo autor.}
    \end{figure}
    
\end{itemize}

\section{Hospedagem e custos}\label{cap:host}

O sistema foi implantado em um ambiente de produção utilizando serviços de hospedagem em nuvem. A aplicação NextJS foi hospedada na plataforma Vercel\footnote{\url{https://vercel.com/}}. O servidor de sockets foi hospedado na plataforma Railway\footnote{\url{https://railway.app/}}. O banco de dados foi hospedado na plataforma ElephantSQL\footnote{\url{https://www.elephantsql.com/}}. A implantação foi realizada sem maiores problemas e o sistema está disponível e funcionando como o esperado. A Tabela \ref{tab:Tab_1} exibe algumas faixas de precificação para os serviços utilizados. No momento o sistema opera com os planos gratuitos.

\begin{table}[htb]
	\ABNTEXfontereduzida
	\caption{\label{tab:Tab_1}Preços dos planos de hospedagem em nuvem.}
 \begin{center}
	\begin{tabular}{@{}p{2.0cm}p{1.5cm}p{3.0cm}p{3.5cm}@{}}
		\toprule
		\textbf{Plataforma} & \textbf{Plano} & \textbf{Preço} & \textbf{Limites} \\ \midrule
		Vercel & Hobby & \ Gratuito & Projetos pessoais ou não comerciais. Deploy a partir do CLI ou integrações git. HTTPS/SSL automático. \\
		Vercel & Pro & \$20 por membro da equipe por mês. & 1TB de banda. 1000 GB/horas de execução. \\
		Vercel & Enterprise & Personalizado. & Infraestrutura de build isolada, em hardware melhor, sem filas. \\
		Railway & Starter & Gratuito & \$5.00 em créditos de recursos todos os meses com tempo de execução de 500 horas. \\
		Railway & Hobby & \$10 em créditos por mês & 8 GB de RAM / 8 vCPU por serviço \\
		Railway & Pro & \$20 por mês & Acesso compartilhado ao espaço de trabalho para equipes. \\
		Railway & Enterprise & Personalizado & Personalizado \\
		ElephantSQL & Tiny & Gratuito & 20 MB, 5 conexões concorrentes \\
		ElephantSQL & Simple & \$5 por mês & 500 MB, 10 conexões concorrentes \\
            ElephantSQL & Enormous & \$199 por mês & 250 GB, centenas de conexões concorrentes \\
  \bottomrule
	\end{tabular}
 \end{center}
	\fonte{Sites das respectivas plataformas.}
\end{table}

A hospedagem em nuvem permite a escalabilidade do sistema, o que significa que é possível aumentar a capacidade do sistema de acordo com a demanda, sem a necessidade de investimentos em infraestrutura própria.

\section{Considerações finais} \label{cap:final}

Os resultados obtidos a partir da implementação do sistema são satisfatórios e atendem aos objetivos estabelecidos neste trabalho. O sistema desenvolvido é funcional, seguro e escalável, atendendo às necessidades. As tecnologias utilizadas mostraram-se adequadas para o desenvolvimento do sistema para automação de pagamentos e para a integração de um agente de distribuição exemplificado pelo NodeMCU com a aplicação web.


	
	% 4 - Conclusão
	\chapter{Materials and Methods} \label{cap:materials}


% 4. Challenges and Trends
% Sensor sensitivity, calibration, and reliability (e.g., SAW, LiDAR challenges).
% Power management and energy harvesting (e.g., Beat Sensors).
% Communication constraints in remote areas.
% Cybersecurity in embedded WSN devices.
% LMM and AI uses examples AI/ML for anomaly detection, forecasting.
% Edge computing trends.
% Scalability.
% Battery life and energy harvesting.
% Integration with cloud and AI pipelines

% In RF, SAW is fully consolidated. In environmental and chemical sensing, it is still a highly active research area with limited commercialization. attempt to resume: ”SAW shows high potential for environmental monitoring, especially for miniaturized, high-sensitivity applications, but adoption is still mostly at the experimental or niche industrial level.” 


To achieve the proposed objectives in this project, the work is divided into a few phases. In the first phase tests will be conducted in a controlled environment to evaluate the different sensors performance, without interferences form the network connection elements or variations in environmental conditions. In this phase the sensors will be integrated into a microcontroller-based sensor node and classified to define the best sensor for the application in monitoring a large river, such as the Itajaí-Açu river, and the most cost efficient solution for a smaller channel/river. In this phase the sensors will be tested, with the goal of evaluating their performance in terms of accuracy, precision, range, resolution, and susceptibility to common interferences such as temperature shifts, target surface variations, and water conditions. 

On a second phase, the sensor nodes will be connected to a \gls{LoRa} / \gls{LoRaWAN} network, and the data will be transmitted to a backend database for storage and visualization. The network will be designed to operate in a star topology, with one central gateway and multiple sensor nodes. The sensor nodes will be programmed to periodically make multiple readings on the distance data, process it and send  to the gateway using \gls{LoRa} modulation.

Finally, the system will be deployed in a real environment, such as a river or water body, to evaluate its performance in terms of data accuracy, transmission success rate, energy efficiency, and resilience against ambient environmental conditions. The collected data will be analyzed to provide insights into the performance of the sensors and the overall system.

\section{System Architecture}
The proposed monitoring system is built around a modular \gls{WSN} designed for remote water level monitoring. Each node in the network is equipped with non-contact distance sensors, \gls{LiDAR} and/or ultrasonic, and a \gls{LoRa}/\gls{LoRaWAN} communication module. The architecture comprises the following subsystems:

\begin{itemize}
\item \textbf{Data Acquisition}: Utilizes \gls{LiDAR} and/or ultrasonic sensors for distance measurement.
\item \textbf{Processing Unit}: A low-power microcontroller (e.g., ESP32, ESP8266) reads sensor data and manages system logic.
\item \textbf{Communication Interface}: Sends processed data via \gls{LoRa}/\gls{LoRaWAN} to a centralized gateway.
\item \textbf{Power Management}: Employs rechargeable batteries and optional solar panels.
\end{itemize}

\section{First Phase: Sensor Testing and Calibration} 

Sensors are mounted on adjustable rigs to test performance against fixed surfaces at various distances and materials. Environmental conditions like ambient temperature and lighting are controlled.

\subsection{Microcontroller Unit}
A low-power microcontroller such as the ESP8266 is selected for this project due to its sufficient performance, integrated Wi-Fi, and support for external LoRa modules. The MCU interfaces with sensors using UART,I2C and digital I/O.

This microcontroller will be used throughout the project, from the first phase of testing the sensors to the final deployment in the field. It will be responsible for reading sensor data, processing it, and transmitting it via LoRa. The microcontroller will also be responsible for managing power consumption, entering a low-power sleep mode when not actively collecting or transmitting data.

\subsection{Sensors}
Four distance sensors were used to capture water level data for performance comparison:
\begin{itemize}
\item \textbf{TF-Luna (\gls{LiDAR})}: Compact, accurate, and low-cost, with up to 8 m range.
\item \textbf{TF-Nova (\gls{LiDAR})}: Offers improved range (up to 12 m) and stability over TF-Luna.
\item \textbf{JSN-SR04T (Ultrasonic)}: Waterproof, outdoor-suitable sensor with 6 m range.
\item \textbf{HC-SR04 (Ultrasonic)}: Basic sensor for initial prototyping, up to 4 m range.
\end{itemize}

The four sensors will be tested in the same conditions, connected to a ESP8266 microcontroller, which will read the data and send it to a connected computer for analysis. After the tests are completed the data will be analyzed to determine the best sensor for the application, considering factors such as cost, performance, and environmental conditions.

The list of tests to be performed was defined based on the sensors specifications, the project requirements and based on some findings of previous works, \cite{paul_2020_a} found that the temperature can affect the readings of the \gls{LiDAR} sensors, and \cite{tamari_2016_flash} mentions that future works should look  to check if the number of suspended small particles in the water could impact the readings of \gls{LiDAR} sensors. The tests will be as follows:

\begin{itemize}
    \item \textbf{Distance Range Test}: Sensors are tested at various distances to detect soft and hard reading limits.
     \item \textbf{Precision Test}: Multiple readings are taken at fixed distances to evaluate precision.
     \item \textbf{Accuracy Test}: Comparing sensor readings with reference measurements to determine accuracy.
     \item \textbf{Light Interference Test}: Sensors are tested under different lighting conditions to assess sensitivity to ambient light.
     \item \textbf{Surface Material Test}: Sensors are tested on different surfaces (clean water, water mixed with earth, solid materials).
     \item \textbf{Temperature Variation Test}: Sensors are tested at different temperatures.
     \item  \textbf{Environmental Interference Test}: Sensors are tested in simulated and controlled envirmental conditions with wind, rain, and other environmental factors.
\end{itemize}

\section{Second Phase: System Integration and Communication}

After evaluating the sensors, the selected ones are integrated into a sensor node. Besides the sensor, the node includes the LoRa transceiver, a microcontroller, and a power management system.  The node is programmed to read sensor data, process it, and transmit it via LoRa. The node is designed to operate in low-power mode, waking up at set intervals to collect and send data.

\subsection{Communication Module}
A LoRa transceiver operating at 915 MHz enables long-range data transmission with minimal energy consumption. Each node sends data to a gateway(LoRa Receiver) which forwards it to a backend database for storage and visualization. Figure \ref{fig:node_integration_diagram} illustrates the integration of the LoRa nodes and the receiver.

\begin{figure}[h]
    \centering
    \caption{LoRa Nodes and Receiver integration diagram.}
    \includegraphics[width=0.65\textwidth]{figuras/lora_nodes_and_receiver.PNG}
    \fonte{The author.}
    \label{fig:node_integration_diagram}
\end{figure}

\section{Third Phase: Field Deployment and Evaluation}

\subsection{Field Deployment}
The complete system is deployed near a river or water body. Each node is securely housed and positioned above the water surface.
\begin{itemize}
\item Evaluation criteria: data accuracy, transmission success rate, energy efficiency, and resilience.
\item Nodes report data every 15 minutes; timestamps allow trend tracking.
\end{itemize}

\subsection{Node Software Architecture}
The software architecture is designed to ensure efficient data acquisition, processing, and transmission. The main loop reads sensor data, processes it, and transmits packets via LoRa only in set intervals to conserve energy, entering a deep sleep state otherwise. The flow of the node while it is active is descrbed in Figure \ref{fig:sensor_node_software_diagram}.	

\begin{figure}[h]
	\centering
	\caption{Sensor node diagram.}
	\includegraphics[width=0.65\textwidth]{figuras/sensor_node_software_diagram.PNG}
	\fonte{The author.}
	\label{fig:sensor_node_software_diagram}
\end{figure}

\subsection{Data Communication and Network Topology}
Each node periodically collects distance data and sends it to the gateway using LoRa modulation. The gateway relays this data to a server through Wi-Fi or cellular connection. The network is structured in a star topology with one central gateway and multiple sensor nodes.

\section{Data Processing and Storage}
Received sensor data is stored in a cloud database for further analysis. Time-series data is cleaned and visualized using statistical and signal processing tools. Metrics such as mean, standard deviation, and RMSE are computed.

\section{Expected Outcomes}
This integrated system aims to evaluate the accuracy, stability, and cost-effectiveness of multiple distance sensors in real-world hydrological monitoring. The LoRa-based architecture enables remote deployment and real-time water level reporting, supporting early warning and sustainability goals. 

\begin{figure}[h]
	\begin{lstlisting}[caption={Exemplo de um schema no Prisma.}, label={code:prisma}]
	generator client {
		provider = "prisma-client-js"
	}
	
	datasource db {
		provider = "postgresql"
		url      = env("DATABASE_URL")
	}
	
	model User {
		id        String        @id @unique @default(uuid())
		email     String        
		password  String
		cpf_cnpj  String        @unique
		name      String
		role      RoleEnumType? @default(user)
		createdAt DateTime      @default(now())
		updatedAt DateTime      @updatedAt
	}
	
	enum RoleEnumType {
		user
		admin
	}
		
	\end{lstlisting}
	
	\fonte{The author.}
	\end{figure}


\begin{table}
	\ABNTEXfontereduzida
	\caption{\label{tab:Tab_2}Table comparing sensors.}
 \begin{center}
	\begin{tabular}{@{}p{2.0cm}p{2cm}p{2.5cm}p{2cm}p{2.0cm}@{}}
		\toprule
		\textbf{Sensor} & \textbf{Type} & \textbf{Range} & \textbf{Precision} & \textbf{Accuracy} \\ \midrule
		JNS SR04T & Ultrasonic & 20 cm a 4 m & ±1 cm  & ±1 cm \\
		HC-SR04 & Ultrasonic & 2 cm a 4 m & ±3 mm  & ±3 mm \\
		TF-Luna & \gls{LiDAR} & 0,2 m a 8 m & ±1 cm   & ±1 cm \\
		TF-Nova & \gls{LiDAR} & 0,1 m a 12 m & ±1 cm  & ±1 cm \\
  \bottomrule
	\end{tabular} 
 \end{center}
	\fonte{The author.}
	\label{tab:comparativo}
\end{table}

	
	
	% Elementos pós-textuais
	\postextual
	
	
	% Referências bibliográficas
	\begingroup
	    \printbibliography[title=REFERÊNCIAS]
	\endgroup
	
	
	%Reconfiguração do título para apêndices e anexos
	 \renewcommand{\ABNTEXchapterupperifneeded}[1]{#1} 
	\makeatletter
	\settocpreprocessor{chapter}{%
      \let\tempf@rtoc\f@rtoc%
      \def\f@rtoc{%
      \texorpdfstring{{\tempf@rtoc}}{\tempf@rtoc}}%
      }
    \makeatother
	
	
	% Apêndices
    %\begin{apendicesenv}
    	%\partapendices* 
    	%\input{pos_textual/apendice_a}
    %\end{apendicesenv}

    % Anexos
    %\begin{anexosenv}
    	%\partanexos*
    	%\input{pos_textual/anexo_a}
    %\end{anexosenv}

\end{document}