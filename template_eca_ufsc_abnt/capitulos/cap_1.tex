% ----------------------------------------------------------
\chapter{Introdução} \label{cap:intro}
% ----------------------------------------------------------
% Motivation for environmental monitoring
% AI Increases the need for data, which can be supplied from sensors etc.
% Importance of Wireless Sensor Networks (WSNs) in environmental monitoring.
% Increasing relevance of low-power, long-range technologies like LoRa.
% Categorization: Water, Soil, and Air/Chemical monitoring.
% Smart cities?
% Outline of paper structure.

A crescente urbanização e as mudanças climáticas causam impactos diversos em diferentes camadas da sociedade, ameaçando individuos em situaçoes de risco em catastrofes como enchentes, ou causando impactos na produção agricola com mudanças climáticas \cite{jonkman_2005_global, hall_2014_understanding, bragana_2024_anlise, borga_2014_hydrogeomorphic}. Esses fenômenos demonstram a necessidade de sistemas eficazes de monitoramento que possam fornecer mais dados sobre as condições ambientais, e nos ajudem a monitorar, analisar e prever  esses fenômenos \cite{hall_2014_understanding, lin_2020_semantic, lo_2015_visual, iqbal_2021_how}.

Quando falamos de monitoramento do ambiente, falamos de uma ampla gama de aplicações e dispositivos, especialmente em regiões remotas e de difícil acesso, representa um desafio técnico significativo. A extensão territorial dessas áreas, aliada a condições ambientais adversas e à crescente demanda por dados em tempo real, exige soluções tecnológicas que sejam robustas, econômicas e escaláveis \cite{chen_2013_natural, yellampalli_2021_wireless, pule_2017_wireless}.

Nesse contexto, as Redes de Sensores Sem Fio (WSNs) têm se consolidado como uma ferramenta promissora, oferecendo vantagens significativas em termos de flexibilidade de implantação, capacidade de resposta rápida e custo-benefício em comparação com infraestruturas tradicionais de monitoramento. As WSNs permitem amostragem espacial densa e coleta contínua de dados, aspectos críticos para o acompanhamento de fenômenos naturais \cite{chen_2013_natural, ferreira_2023_conception, pule_2017_wireless}.

A necessidade de monitoramento ambiental em tempo real têm ampliado a demanda por dados, ressaltando a importância de tecnologias de baixo consumo energético e longo alcance, como o LoRa. Essas tecnologias são especialmente adequadas para aplicações em áreas remotas, onde a infraestrutura de comunicação é limitada ou inexistente \cite{pule_2017_wireless, chen_2013_natural, ferreira_2023_conception}.

Além disso, vemos uma tendência crescente na demanda de dados para integração com sistemas de inteligência artificial, que podem extrair analises e tendências valiosas e auxiliar na tomada de decisões em tempo real. Essa integração potencializa a capacidade preditiva dos sistemas de monitoramento ambiental e abre oportunidades para novas aplicações \cite{nr_2025_ai,mukhopadhyay_2021_artificial,ferreira_2023_conception,chen_2013_natural, lin_2020_semantic}. Em 2021 foi sugerida uma nova definição para este conceito, unindo AI e IoT como AIoT, que se refere à integração de inteligência artificial com a Internet das Coisas (IoT) para criar sistemas mais inteligentes e autônomos \cite{mukhopadhyay_2021_artificial}.

Esta revisão da literatura está estruturada em cinco seções principais. Primeiramente, serão abordados os sensores e as tecnologias utilizadas no monitoramento ambiental, categorizando-os conforme os domínios de aplicação (hidrológico, solo e qualidade do ar). Em seguida, será discutido o papel das Redes de Sensores Sem Fio e das tecnologias de comunicação associadas. Posteriormente, serão apresentados os principais desafios e tendências atuais no campo do monitoramento ambiental, com foco em inovações tecnológicas. Por fim, serão expostas as conclusões e perspectivas futuras.

Embora revisões anteriores tenham abordado de forma aprofundada sensores para o monitoramento do solo como as realizadas por \textcite{yin_2021_smart, teng_2014_soil, queiroz_2020_sensors} ou técnicas de medição de nível de água como as feitas por \textcite{mohindru_2023_development, nr_2025_ai, wu_2023_a}, observa-se uma lacuna de estudos recentes que integrem, de maneira abrangente, tanto as tecnologias de sensores quanto as infraestruturas de comunicação, trazendo conceitos emergents no monitoramento ambiental e não se limitando a aplicações específicas.

Esta revisão busca preencher essa lacuna ao oferecer uma visão que contemple as tecnologias de sensores aplicadas ao monitoramento de solo, água e/ou ar/compostos químicos, acompanhada de uma análise das tecnologias de comunicação sem fio, estratégias de eficiência energética e desafios de segurança em implantações de WSNs para o monitoramento ambiental. A principal contribuição deste trabalho consiste em consolidar avanços multidisciplinares, destacar tecnologias ainda pouco exploradas e fornecer uma análise comparativa que possa orientar futuras pesquisas e aplicações no desenvolvimento de projetos para monitoramento ambiental.
