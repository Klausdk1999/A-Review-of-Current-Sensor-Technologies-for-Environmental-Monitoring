% ----------------------------------------------------------
\chapter{Introdução} \label{cap:intro}
% ----------------------------------------------------------

The increasing severity of climate change, pollution, and ecological degradation has elevated the need for continuous, accurate, and scalable environmental monitoring. Understanding the dynamics of air quality, water purity, soil health, and meteorological conditions is essential for informed decision-making in fields such as public health, agriculture, urban planning, and conservation. Traditional environmental data collection methods—largely manual and periodic—are no longer sufficient to capture the spatial and temporal complexity of these phenomena.

Recent advancements in sensor technology, coupled with the rise of the Internet of Things (IoT) and wireless sensor networks (WSNs), have revolutionized the way environmental data is collected, transmitted, and analyzed. These systems enable real-time, distributed, and often autonomous monitoring across vast geographic areas, even in remote or harsh environments. The convergence of low-power sensors, edge computing, and long-range wireless communication protocols has made it feasible to deploy scalable and cost-effective monitoring infrastructures.

This review aims to provide a comprehensive overview of the current state of sensor technologies used in environmental monitoring. It explores the different types of environmental parameters and the corresponding sensor modalities, examines how these sensors are integrated into modern IoT-based systems, and discusses the challenges related to data quality, calibration, power consumption, and communication. Additionally, it highlights emerging trends in sensor development and identifies research opportunities in this rapidly evolving field.

\section{Objetivo Geral}

O objetivo geral deste projeto é avaliar o desempenho de sensores ultrassônicos, sensores LiDAR e câmeras na medição do nível de rios, comparando suas capacidades e limitações em diferentes cenários. Com base nos resultados, busca-se determinar qual tecnologia apresenta melhor custo-benefício e maior adequação para essa aplicação específica.
Foram escolhidos três tipos de sensores para análise: sensores ultrassônicos, sensores LiDAR e câmeras.

\section{Objetivos Específicos}

\begin{enumerate}

    \item Realizar uma revisão bibliográfica sobre as principais tecnologias de sensores utilizadas no monitoramento de níveis de líquidos e corpos d'água.
    
    \item Definir um conjunto de critérios para avaliação dos sensores, considerando aspectos como precisão, alcance máximo, resistência a interferências ambientais e custo.

    \item Conduzir testes experimentais com sensores ultrassônicos, sensores LiDAR e câmeras em ambiente controlado.

    \item Realizar testes em um cenário real, avaliando o impacto de fatores como variação climática, presença de detritos e mudanças no fluxo da água.

    \item Comparar os resultados obtidos e identificar as aplicações mais adequadas para cada tipo de sensor.

    \item Determinar a tecnologia mais eficiente para o monitoramento do nível de rios, considerando aspectos de viabilidade técnica e econômica.

    \item Elaborar recomendações para a implementação de sistemas de medição baseados nos sensores analisados.

\end{enumerate}

\section{Estrutura do documento}

Este trabalho está dividido em seis capítulos. O Capítulo \ref{cap:intro} apresenta a introdução e os objetivos da pesquisa. O Capítulo \ref{cap:fundamentacao} trata da fundamentação teórica, abordando os princípios de funcionamento dos sensores analisados. O Capítulo \ref{cap:arquitetura} descreve a metodologia e os critérios adotados para a realização dos testes. O Capítulo \ref{cap:desenvolvimento} apresenta os experimentos conduzidos e as tecnologias utilizadas. O Capítulo \ref{cap:resultados} discute os resultados obtidos e compara as diferentes abordagens. Por fim, o Capítulo \ref{cap:conclusao} traz as considerações finais e sugestões para trabalhos futuros.

