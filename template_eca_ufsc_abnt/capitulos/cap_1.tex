% ----------------------------------------------------------
\chapter{Introdução} \label{cap:intro}
% ----------------------------------------------------------

O monitoramento do nível de rios é uma atividade essencial para a gestão de recursos hídricos, prevenção de enchentes e planejamento ambiental. Tecnologias baseadas em sensores vêm sendo amplamente utilizadas para essa finalidade, permitindo medições automáticas e em tempo real. No entanto, a escolha do sensor mais adequado depende de diversos fatores, como a precisão da medição, o alcance máximo, as condições ambientais suportadas e os custos de implementação.

Entre as principais tecnologias utilizadas para essa aplicação, destacam-se os sensores ultrassônicos, sensores LiDAR e câmeras. Cada um desses dispositivos apresenta vantagens e limitações que impactam diretamente sua viabilidade para medições hidrológicas. Sensores ultrassônicos, por exemplo, são comumente utilizados devido ao seu baixo custo e simplicidade de instalação, mas podem ser afetados por variações de temperatura e umidade. Sensores LiDAR oferecem maior precisão e alcance, porém apresentam custo elevado e maior consumo energético. Câmeras, por sua vez, podem fornecer medições indiretas baseadas em visão computacional, mas sua eficácia depende da qualidade da imagem e da presença de obstruções no campo de visão.

Neste contexto, o presente trabalho propõe a realização de uma análise comparativa entre esses três tipos de sensores, a fim de avaliar seu desempenho em diferentes condições ambientais e determinar sua aplicabilidade no monitoramento do nível de rios. Os testes serão conduzidos em ambiente controlado e em um cenário real, analisando aspectos como precisão, estabilidade da medição, interferências externas e facilidade de instalação. A partir dos resultados obtidos, será possível indicar as aplicações mais adequadas para cada tecnologia e definir a solução mais eficiente para o monitoramento hidrológico.

\section{Objetivo Geral}

O objetivo geral deste projeto é avaliar o desempenho de sensores ultrassônicos, sensores LiDAR e câmeras na medição do nível de rios, comparando suas capacidades e limitações em diferentes cenários. Com base nos resultados, busca-se determinar qual tecnologia apresenta melhor custo-benefício e maior adequação para essa aplicação específica.
Foram escolhidos três tipos de sensores para análise: sensores ultrassônicos, sensores LiDAR e câmeras.

\section{Objetivos Específicos}

\begin{enumerate}

    \item Realizar uma revisão bibliográfica sobre as principais tecnologias de sensores utilizadas no monitoramento de níveis de líquidos e corpos d'água.
    
    \item Definir um conjunto de critérios para avaliação dos sensores, considerando aspectos como precisão, alcance máximo, resistência a interferências ambientais e custo.

    \item Conduzir testes experimentais com sensores ultrassônicos, sensores LiDAR e câmeras em ambiente controlado.

    \item Realizar testes em um cenário real, avaliando o impacto de fatores como variação climática, presença de detritos e mudanças no fluxo da água.

    \item Comparar os resultados obtidos e identificar as aplicações mais adequadas para cada tipo de sensor.

    \item Determinar a tecnologia mais eficiente para o monitoramento do nível de rios, considerando aspectos de viabilidade técnica e econômica.

    \item Elaborar recomendações para a implementação de sistemas de medição baseados nos sensores analisados.

\end{enumerate}

\section{Estrutura do documento}

Este trabalho está dividido em seis capítulos. O Capítulo \ref{cap:intro} apresenta a introdução e os objetivos da pesquisa. O Capítulo \ref{cap:fundamentacao} trata da fundamentação teórica, abordando os princípios de funcionamento dos sensores analisados. O Capítulo \ref{cap:arquitetura} descreve a metodologia e os critérios adotados para a realização dos testes. O Capítulo \ref{cap:desenvolvimento} apresenta os experimentos conduzidos e as tecnologias utilizadas. O Capítulo \ref{cap:resultados} discute os resultados obtidos e compara as diferentes abordagens. Por fim, o Capítulo \ref{cap:conclusao} traz as considerações finais e sugestões para trabalhos futuros.

