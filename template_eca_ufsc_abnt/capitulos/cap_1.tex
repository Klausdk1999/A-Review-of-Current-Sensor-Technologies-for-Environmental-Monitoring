% ----------------------------------------------------------
\chapter{Introdução} \label{cap:intro}
% ----------------------------------------------------------
% Motivation for environmental monitoring
% AI Increases the need for data, which can be supplied from sensors etc.
% Importance of Wireless Sensor Networks (WSNs) in environmental monitoring.
% Increasing relevance of low-power, long-range technologies like LoRa.
% Categorization: Water, Soil, and Air/Chemical monitoring.
% Smart cities?
% Outline of paper structure.

O monitoramento de sistemas ambientais, especialmente em regiões remotas e de difícil acesso, representa um desafio técnico significativo. A extensão territorial dessas áreas, aliada a condições ambientais adversas e à crescente demanda por dados em tempo real, exige soluções tecnológicas que sejam robustas, econômicas e escaláveis.

Nesse contexto, as Redes de Sensores Sem Fio (WSNs) têm se consolidado como uma ferramenta promissora, oferecendo vantagens significativas em termos de flexibilidade de implantação, capacidade de resposta rápida e custo-benefício em comparação com infraestruturas tradicionais de monitoramento. As WSNs permitem amostragem espacial densa e coleta contínua de dados, aspectos críticos para o acompanhamento de fenômenos naturais \cite{chen_2013_natural, ferreira_2023_conception, pule_2017_wireless}.

O avanço da inteligência artificial e a necessidade de monitoramento ambiental em tempo real têm ampliado a demanda por dados de alta qualidade, ressaltando a importância de tecnologias de baixo consumo energético e longo alcance, como o LoRa. Essas tecnologias são especialmente adequadas para aplicações em áreas remotas, onde a infraestrutura de comunicação é limitada ou inexistente \cite{pule_2017_wireless, chen_2013_natural, ferreira_2023_conception}.

Esta revisão da literatura está estruturada em cinco seções principais. Primeiramente, serão abordados os sensores e as tecnologias utilizadas no monitoramento ambiental, categorizando-os conforme os domínios de aplicação mais comuns (hidrológico, solo e qualidade do ar). Em seguida, será discutido o papel das Redes de Sensores Sem Fio e das tecnologias de comunicação associadas. Posteriormente, serão apresentados os principais desafios e tendências atuais no campo do monitoramento ambiental, com foco em inovações tecnológicas. Por fim, serão expostas as conclusões e perspectivas futuras.

Embora revisões anteriores tenham abordado de forma aprofundada sensores para o monitoramento do solo [Ref] ou técnicas de medição de nível de água [Ref], observa-se uma lacuna de estudos que integrem, de maneira abrangente, tanto as tecnologias de sensores quanto as infraestruturas de comunicação, como o LoRa, além de abordagens emergentes, como a comunicação baseada em intervalos (beat-based communication). Além disso, tecnologias de sensores emergentes, como os sensores baseados em ondas acústicas de superfície (SAW), apesar de frequentemente discutidas de forma isolada, ainda não foram avaliadas de maneira consistente em diferentes domínios ambientais.

Esta revisão busca preencher essa lacuna ao oferecer uma visão abrangente que contemple as tecnologias de sensores aplicadas ao monitoramento de solo, água e ar/compostos químicos, acompanhada de uma análise das tecnologias de comunicação sem fio, estratégias de eficiência energética e desafios de segurança em implantações de WSNs para o monitoramento ambiental. A principal contribuição deste trabalho consiste em consolidar avanços multidisciplinares, destacar tecnologias ainda pouco exploradas e fornecer uma análise comparativa que possa orientar futuras pesquisas e aplicações.
