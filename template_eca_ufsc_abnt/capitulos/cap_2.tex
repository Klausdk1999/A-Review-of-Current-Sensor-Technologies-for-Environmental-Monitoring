
% 2 Sensors and Technologies for Environmental Monitoring
% Sensor classification from that “sensors daily use…” - Active or passive and applications.
% 2.1 Water Monitoring
% Focus: Water level (ultrasonic, LiDAR, float sensors),Water quality (pH, turbidity, EC sensors), flood detection, river monitoring.
% Innovations: Archimedes-based sensors, object recognition methods, SAW?.

% Relevant Articles:
% 1"An Intelligent Water Level Estimation System" — Uses object recognition via deep learning for water levels. (Observations: Estimation via camera + ML.)
% 2"A Technical Evaluation of Lidar-Based Measurement of River Water Levels" — Discusses LiDAR for precise water monitoring. (High relevance.)
% 3"A low-cost ultrasonic sensor for online monitoring of liquid level" — Simple ultrasonic-based approach. (Low relevance but practical example.)
% 4"SAW Sensor based a Novel Hydrostatic Liquid Level Sensor" — Novel SAW-based liquid level sensor. (Observation: Hydrostatic pressure with SAW.)
% 5"Modeling and testing of a highly sensitive surface acoustic wave hydrostatic liquid level sensor" — Focus on SAW tech for water levels.
% 6"Development of liquid level measurement technologies: A review" — Industrial-focused review of level sensors.

% 2.2 Soil Monitoring
% Focus: Soil moisture, nutrients, agriculture precision.
% Use of SAW sensors, RFID-based systems, and nanotechnology.
% Relevant Articles:
% 1"Smart Agriculture Systems: Soil Sensors and Plant Wearables for Smart and Precision Agriculture" — Extensive review, strong base for soil monitoring section. (MIT paper, highly relevant.)
% 2"Conception and Design of WSN Sensor Nodes Based on Soil Moisture Monitoring" — Focus on soil pH, turbidity, and temperature sensors. (Observation: Detailed sensor design.)
% 3"Applications of Sensor Networks and Remote Sensing in Precision Agriculture" — General but with some relevance to WSN in agriculture. (Observation: Somewhat shallow but covers multiple angles.)

% 2.3 Air and Chemical Monitoring
% Focus: Air quality, gas detection, environmental chemical monitoring.
% "SAW Sensors for Chemical Vapors and Gases" — Discusses SAW sensor principles for air/gas detection. (Observation: Emphasizes SAW advantages like sensitivity and robustness.)
% "A Progress Review on Solid‐State LiDAR and Nanophotonic Sensors" — Discusses LiDAR, photonic sensors applicable to particulate air detection or similar use cases.

\chapter{Sensores e Tecnologias para Monitoramento Ambiental} \label{cap:sensors}

Nosso dia a dia é repleto de sensores, sempre coletando dados sobre o ambiente ao nosso redor. Esses sensores podem ser classificados de muitas formas diferentes, alguns exemplos de categorização relevantes para entendimento são: sensores ativos e passivos, onde os sensores ativos emitem algum tipo de sinal para medir uma grandeza física, enquanto os sensores passivos apenas captam sinais já existentes no ambiente \cite{javaid_2021_sensors}. Outra forma de classificar os sensores é a forma com que interagem com o ambiente, onde sensores de contato medem grandezas físicas diretamente em contato com o meio, enquanto sensores de não contato medem grandezas físicas sem necessidade de contato direto \cite{javaid_2021_sensors, mohindru_2023_development, wu_2023_a}. Por fim, existem ainda muitas outras formas de classificar sensores, mas nesse trabalho optou-se pela classificação de sensores de acordo com o tipo de aplicação, como sensores de nível de água, sensores de qualidade do ar, sensores de temperatura, de forma a melhor guiar pesquisas nesses ambientes. 


\section{Monitoramento de Água}\label{sec:water_monitoring}

Quando falamos de monitoramento de água, podemos dividir os sensores em dois tipos principais: sensores de nível de água e sensores de qualidade da água. Os sensores de nível de água são utilizados para medir a altura da coluna d'água em corpos hídricos, como rios, lagos e reservatórios ou em ambientes controlados como tanques e poços. Já os sensores de qualidade da água são usados para medir parâmetros como pH, turbidez, condutividade elétrica (EC) e outros compostos químicos presentes na água.

\subsection{Sensores de Nível de Água}\label{subsec:water_level_sensors}

Para o monitoramento de nível de água em ambientes naturais como rios e lagos, podemos destacar os tradicionais sensores de contato, como os sensores de pressão, que medem a pressão exercida pela coluna d'água sobre o sensor, um destes foi testado por \textcite{santana_2024_development}, destacando que o mesmo faz boas leituras, sendo indiferente a parametros como turbidez da água, porém o mesmo trabalho destaca o ponto fraco desta tecnologia que é a sua aplicação em ambientes hostis, visto que em situaçoes como a cheia do rio, onde há uma grande quantidade de entulhos e forte correnteza, esses sensores podem ser comprometidos. Além disso sensores de contato para monitoramento em ambientes naturais podem envolver conjuntos de boias, que detecam a altura da água em relação a um ponto fixo, ou sensores de flutuação, que utilizam um flutuador conectado a um cabo ou corda para medir a altura da água. Esses sensores são simples e eficazes, mas podem ser afetados por fatores como detritos flutuantes, variações de temperatura e corrosão \cite{mohammadrezamasoudimoghaddam_2024_a,santana_2024_development,paul_2020_a,yukawa_2025_an} .

Quando buscamos sensores para monitoramento de nível em ambientes controlados como tanques e poços, e não estamos preocupados em sensores de longo alcance, não precisamos nos preocupar tanto com fatores como detritos flutuantes, e podemos utilizar os sensores ja mencionados anteriormente para aplicações em rios e lagos, além de outros metodos mais simples como os sensores de nível de água resistivos, que medem a resistência elétrica entre dois eletrodos submersos na água. Esses sensores são econômicos e fáceis de instalar, mas podem ser afetados por corrosão e depósitos minerais \cite{santana_2024_development, mohindru_2023_development}. Para essa categoria existem ainda tecnologias emergentes que demonstram potencial, como os sensores passivos de nível de água baseados em ondas acústicas, que utilizam a variação da velocidade do som na água para medir a altura da coluna d'água, como os desenvolvidos por \textcite{ali_2020_saw} e \textcite{sreejith_2024_modeling}. Outra tecnologia que aparece como promissora são os sensores de nível com fibra óptica, se baseiam no principio de Arquimedes da hidrostatica e utilizam a variação da luz transmitida através de uma fibra óptica e um elemento flutuante para medir a altura da coluna d'água, o principio é demonstrado com avanços desenvolvidos por \textcite{ramos_2025_high}. Esses sensores são altamente precisos, mas ainda são complexos de instalar e operar.

Para aplicações de monitoramento sem contato, esta revisão identificou tres tecnologias emergentes, sensores ultrassônicos, sensores lidar, e monitoramento remoto através de imagens geradas por aplicações aeroespaciais.

Os sensores de nível de água baseados em ultrassom, emitem ondas sonoras e medem o tempo que essas ondas levam para retornar ao sensor. Esses sensores são amplamente utilizados devido à sua precisão e capacidade de operar em ambientes com variações de temperatura e pressão \cite{mohammadrezamasoudimoghaddam_2024_a, pereira_2022_evaluation}. Por exemplo, o modelo de sensor ultrassônico GY-Us42 foi testado com resultados indicando que o erro médio do dispositivo é inferior a 3\% \cite{mohammadrezamasoudimoghaddam_2024_a}. Outro modelo, o HC-SR04, também foi avaliado como uma alternativa técnica e econômica viável para monitoramento de níveis de água \cite{pereira_2022_evaluation}, além de ser uma boa opção para educação, ciência cidadã e pesquisa devido ao seu baixo custo \cite{bresnahan_2023_a}.

Os sensores LiDAR faz uso de ondas ópticas para medir distâncias e velocidades, sendo amplamente utilizados em metrologia, monitoramento ambiental, arqueologia e robótica. O princípio de medição do LiDAR baseia-se na rugosidade da superfície refletora para gerar reflexão não especular (ou seja, dispersão) do feixe laser emitido.  A luz próxima do infravermelha  (NIR) é a mais comumente utilizada para esse fim, geralmente em comprimentos de onda entre 900 e 1100 nm (270-330 THz), devido ao baixo custo dos lasers operando nessa faixa e à menor densidade de energia em comparação com o espectro visível \cite{li_2022_a, fernandezdiaz_2014_early, smart_2009_river, behroozpour_2017_lidar}. Assim como nos sensores ultrassônicos, a medição do LiDAR é feita pelo tempo de voo (TOF), mas existem dois métodos principais de medição, o TOF pulsado e o AMCW TOF, que diferem na forma como o sinal é emitido e recebido. No TOF pulsado, um pulso óptico é emitido e o tempo que leva para retornar ao sensor é medido, enquanto no AMCW TOF, uma onda contínua modulada em amplitude é utilizada, e a diferença de fase entre os sinais transmitidos e recebidos é medida para determinar a distância \cite{li_2022_a}.
Esses foram explorados como uma alternativa de baixo custo para medir níveis de água a partir de pontes, com testes em laboratório e campo indicando uma boa precisão com erro de 0.1\%, mas com variações significativas causadas pela temperatura do sensor e pela rugosidade da água \cite{paul_2020_a}. Sensores LiDAR instalados em margens de rios para monitoramento de enchentes também foram testados com bons resultados, indicando que diferentes quantidades de pequenas partículas suspensas na água poderiam impactar os resultados, e que o sensor poderia ser usado para monitorar essas partículas suspensas \cite{tamari_2016_flash}. Outro estudo comparou o modelo de sensor LiDAR TF-mini com sensores de pressão linigráficos, mostrando os benefícios do método de medição sem contato do LiDAR em comparação com o método de contato do linigráfico e validando o LiDAR como uma excelente escolha entre as tecnologias de medição de nível de fluido comparadas \cite{santana_2024_development}.


Os sensores de nível de água baseados em imagens, como os utilizados por "RUSSOS DOIDOS?" \textcite{santana_2024_development}, utilizam técnicas de reconhecimento de objetos via aprendizado profundo para estimar o nível da água. Esses sensores são promissores, pois podem ser integrados a sistemas de monitoramento remoto e oferecem uma alternativa não invasiva para medir níveis de água em ambientes naturais e controlados.


% --------- General LORA REVIEW MENTIONS ------  GENERAL FLOOD REVIEW MENTIONS ------ ?? maybe put on fundaments chapter

% Previously ultrasonic was defined as viable option to monitor water levels in river and channels using the sensor model GY-Us42, with results indicating that the device's average error is below 3\% \cite{mohammadrezamasoudimoghaddam_2024_a}. Other work concludes that the ultrasonic sensor model HC-SR04 is a technically and economically viable alternative to monitor water levels \cite{ pereira_2022_evaluation}, and the same sensor model is also praised as a good option for education, citizen science, and research due to low cost \cite{bresnahan_2023_a}. 

% In previous research, LiDAR was explored as a cost-effective sensor for measuring water leves from bridges, with lab and field tests indication a good accuracy with 0.1\% error, but significant variations caused by sensor temperature and water roughness \cite{paul_2020_a}. LiDAR sensors installed on river banks for flash flood monitoring were also tested with good results, and indication that different amounts of small suspended particles on water could impact the results, and that the sensor could also be used to monitor these suspended particles \cite{tamari_2016_flash}. Another study explored how the LiDAR sensor model TF-mini compared to linigraph pressure sensors, showing the benefits of a non-contact measurement method of the LiDAR compared to a contact one with the linigraph and validating the LiDAR as an excellent choice among fluid level measurement technologies compared \cite{santana_2024_development}.

% The contribution of this work is exploring more of the two different methods of non-contact water level measurement, comparing and definig best applications for each one. Adding to it, this work will also explore creating a low-cost, low-power, and long-range water level monitoring system using LoRaWAN technology. The system will be designed to operate in challenging environments, such as flood-prone river systems, where other more traditional monitoring methods may be impractical or too expensive.
