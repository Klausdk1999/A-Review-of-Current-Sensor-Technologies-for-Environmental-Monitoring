
% 2 Sensors and Technologies for Environmental Monitoring
% Sensor classification from that “sensors daily use…” - Active or passive and applications.
% 2.1 Water Monitoring
% Focus: Water level (ultrasonic, LiDAR, float sensors),Water quality (pH, turbidity, EC sensors), flood detection, river monitoring.
% Innovations: Archimedes-based sensors, object recognition methods, SAW?.

% Relevant Articles:
% 1"An Intelligent Water Level Estimation System" — Uses object recognition via deep learning for water levels. (Observations: Estimation via camera + ML.)
% 2"A Technical Evaluation of Lidar-Based Measurement of River Water Levels" — Discusses LiDAR for precise water monitoring. (High relevance.)
% 3"A low-cost ultrasonic sensor for online monitoring of liquid level" — Simple ultrasonic-based approach. (Low relevance but practical example.)
% 4"SAW Sensor based a Novel Hydrostatic Liquid Level Sensor" — Novel SAW-based liquid level sensor. (Observation: Hydrostatic pressure with SAW.)
% 5"Modeling and testing of a highly sensitive surface acoustic wave hydrostatic liquid level sensor" — Focus on SAW tech for water levels.
% 6"Development of liquid level measurement technologies: A review" — Industrial-focused review of level sensors.

% 2.2 Soil Monitoring
% Focus: Soil moisture, nutrients, agriculture precision.
% Use of SAW sensors, RFID-based systems, and nanotechnology.
% Relevant Articles:
% 1"Smart Agriculture Systems: Soil Sensors and Plant Wearables for Smart and Precision Agriculture" — Extensive review, strong base for soil monitoring section. (MIT paper, highly relevant.)
% 2"Conception and Design of WSN Sensor Nodes Based on Soil Moisture Monitoring" — Focus on soil pH, turbidity, and temperature sensors. (Observation: Detailed sensor design.)
% 3"Applications of Sensor Networks and Remote Sensing in Precision Agriculture" — General but with some relevance to WSN in agriculture. (Observation: Somewhat shallow but covers multiple angles.)

% 2.3 Air and Chemical Monitoring
% Focus: Air quality, gas detection, environmental chemical monitoring.
% "SAW Sensors for Chemical Vapors and Gases" — Discusses SAW sensor principles for air/gas detection. (Observation: Emphasizes SAW advantages like sensitivity and robustness.)
% "A Progress Review on Solid‐State LiDAR and Nanophotonic Sensors" — Discusses LiDAR, photonic sensors applicable to particulate air detection or similar use cases.

\chapter{Sensores e Tecnologias para Monitoramento Ambiental} \label{cap:sensors}

Nosso dia a dia é repleto de sensores, sempre coletando dados sobre o ambiente ao nosso redor. Esses sensores podem ser classificados de muitas formas diferentes, alguns exemplos de categorização relevantes para entendimento são: sensores ativos e passivos, onde os sensores ativos emitem algum tipo de sinal para medir uma grandeza física, enquanto os sensores passivos apenas captam sinais já existentes no ambiente \cite{javaid_2021_sensors}. Outra forma de classificar os sensores é a forma com que interagem com o ambiente, onde sensores de contato medem grandezas físicas diretamente em contato com o meio, enquanto sensores de não contato medem grandezas físicas sem necessidade de contato direto \cite{javaid_2021_sensors, mohindru_2023_development, wu_2023_a}. Por fim, existem ainda muitas outras formas de classificar sensores, mas nesse trabalho optou-se pela classificação de sensores de acordo com o tipo de aplicação, como sensores de nível de água, sensores de qualidade do ar, sensores de temperatura, de forma a melhor guiar pesquisas nesses ambientes. 


\section{Monitoramento de Água}\label{sec:water_monitoring}

Quando falamos de monitoramento de água, podemos dividir os sensores em dois tipos principais: sensores de nível de água e sensores de qualidade da água. Os sensores de nível de água são utilizados para medir a altura da coluna d'água em corpos hídricos, como rios, lagos e reservatórios ou em ambientes controlados como tanques e poços. Já os sensores de qualidade da água são usados para medir parâmetros como pH, turbidez, condutividade elétrica (EC) e outros compostos químicos presentes na água.

\subsection{Sensores de Nível de Água}\label{subsec:water_level_sensors}

Para o monitoramento de nível de água em ambientes naturais como rios e lagos, podemos destacar os tradicionais sensores de contato, como os sensores de pressão ou linigráficos, que medem a pressão exercida pela coluna d'água sobre o sensor, um destes foi testado por \textcite{santana_2024_development}, destacando que o mesmo faz boas leituras, sendo indiferente a parametros como turbidez da água, porém o mesmo trabalho destaca o ponto fraco desta tecnologia que é a sua aplicação em ambientes hostis, visto que em situaçoes como a cheia do rio, onde há uma grande quantidade de entulhos e forte correnteza, esses sensores podem ser comprometidos. Além disso sensores de contato para monitoramento em ambientes naturais podem envolver conjuntos de boias, que detecam a altura da água em relação a um ponto fixo, ou sensores de flutuação, que utilizam um flutuador conectado a um cabo ou corda para medir a altura da água. Esses sensores são simples e eficazes, mas podem ser afetados por fatores como detritos flutuantes, variações de temperatura e corrosão \cite{mohammadrezamasoudimoghaddam_2024_a,santana_2024_development,paul_2020_a,yukawa_2025_an} .

Quando buscamos sensores para monitoramento de nível em ambientes controlados como tanques e poços, e não estamos preocupados em sensores de longo alcance, não precisamos nos preocupar tanto com fatores como detritos flutuantes, e podemos utilizar os sensores ja mencionados anteriormente para aplicações em rios e lagos, além de outros metodos mais simples como os sensores de nível de água resistivos, que medem a resistência elétrica entre dois eletrodos submersos na água. Esses sensores são econômicos e fáceis de instalar, mas podem ser afetados por corrosão e depósitos minerais \cite{santana_2024_development, mohindru_2023_development}.

Para essa categoria  monitoramento de nível em ambientes controlados existem ainda tecnologias emergentes que demonstram potencial, como os sensores passivos de nível de água baseados em ondas acústicas, o sensor SAW mede variações de deformação ou pressão na parede do tanque causadas pela mudança no nível da água, convertendo essas alterações em sinais de resposta, como os desenvolvidos por \textcite{ali_2020_saw} e \textcite{sreejith_2024_modeling}. Outra tecnologia que aparece como promissora são os sensores de nível com fibra óptica, se baseiam no principio de Arquimedes da hidrostatica e utilizam a variação da luz transmitida através de uma fibra óptica e um elemento flutuante para medir a altura da coluna d'água, o principio é demonstrado com avanços desenvolvidos por \textcite{ramos_2025_high}. Esses sensores são altamente precisos, mas ainda são complexos de instalar e operar.

Para aplicações de monitoramento sem contato, esta revisão apresenta tres destaques, sensores ultrassônicos, sensores lidar, e monitoramento remoto através de imagens geradas por aplicações aeroespaciais.

Os sensores de nível de água baseados em ultrassom, emitem ondas sonoras e medem o tempo que essas ondas levam para retornar ao sensor. Esses sensores são amplamente utilizados devido à sua precisão e capacidade de operar em ambientes com variações de temperatura e pressão \cite{mohammadrezamasoudimoghaddam_2024_a, pereira_2022_evaluation}. Por exemplo, o modelo de sensor ultrassônico GY-Us42 foi testado com resultados indicando que o erro médio do dispositivo é inferior a 3\% \cite{mohammadrezamasoudimoghaddam_2024_a}. Outro modelo, o HC-SR04, também foi avaliado como uma alternativa técnica e econômica viável para monitoramento de níveis de água \cite{pereira_2022_evaluation}, além de ser uma boa opção para educação, ciência cidadã e pesquisa devido ao seu baixo custo \cite{bresnahan_2023_a}.

Os sensores LiDAR fazem uso de ondas ópticas para medir distâncias e velocidades, sendo amplamente utilizados em metrologia, monitoramento ambiental, arqueologia e robótica\cite{behroozpour_2017_lidar, li_2022_a}. O princípio de medição do LiDAR baseia-se na rugosidade da superfície refletora para gerar reflexão não especular (ou seja, dispersão) do feixe laser emitido. A faixa de luz próxima do infravermelho (NIR) é a mais comumente utilizada para esse fim, geralmente em comprimentos de onda entre 900 e 1100 nm (270-330 THz), devido ao baixo custo dos lasers operando nessa faixa e à menor densidade de energia em comparação com o espectro visível \cite{li_2022_a, fernandezdiaz_2014_early, smart_2009_river, behroozpour_2017_lidar}. Assim como nos sensores ultrassônicos, a medição do LiDAR é feita pelo tempo de voo (TOF), mas existem dois métodos principais de medição, o TOF pulsado e o AMCW TOF, que diferem na forma como o sinal é emitido e recebido. No TOF pulsado, um pulso óptico é emitido e o tempo que leva para retornar ao sensor é medido, enquanto no AMCW TOF, uma onda contínua modulada em amplitude é utilizada, e a diferença de fase entre os sinais transmitidos e recebidos é medida para determinar a distância \cite{li_2022_a}.

Esses sensores foram explorados como uma alternativa de baixo custo para medir níveis de água a partir de pontes, com testes em laboratório e campo indicando uma boa precisão com erro de 0.1\%, mas com variações significativas causadas pela temperatura do sensor e pela rugosidade da água \cite{paul_2020_a}. Sensores LiDAR instalados em margens de rios para monitoramento de enchentes também foram testados com bons resultados, o estudo indicou que partículas suspensas na água impactam positivamente a precisão das leituras, e que o sensor poderia também ser usado para detectar a cncentração dessas partículas suspensas na água\cite{tamari_2016_flash}. Outro estudo comparou o modelo de sensor LiDAR TF-mini com sensores de pressão linigráficos, mostrando os benefícios do método de medição sem contato do LiDAR em comparação com o método de contato do linigráfico e validando o LiDAR como uma excelente escolha entre as tecnologias de medição de nível de fluido comparadas \cite{santana_2024_development}.

Outro método de monitoramento de nível de água em ambientes abertos, e a utilização de monitoramento remoto usando dados obtidos via satélite, como o trabalho realizado por \textcite{jiang_2024_monitoring}, que utiliza imagens e outros dados obtidos via satélite para estimar os níveis de uma bacia hidrográfica ou os de \textcite{ali_2024_satellite} que também usam uma tecnica similar e comparam  estimativas feitas com dados obtidos via satélite com medicoes feitas em varios pontos ao longo do rio com metodos mais tradicionais, ambos os trabalhos apresentam resultados positivos para esta tecnica.

\subsection{Qualidade da Água}
O trabalho de \textcite{ferreira_2023_conception} apresenta um projeto de monitoramento de qualidade da água, que utiliza sensores de pH, turbidez e condutividade elétrica (EC) e integra técnicas de fusão de dados locais e distribuídos com recursos de machine learning para melhorar a detecção de poluentes em tempo real. 

\section{Monitoramento do Solo}
% Focus: Soil moisture, nutrients, agriculture precision.
% Use of SAW sensors, RFID-based systems, and nanotechnology.
% Relevant Articles:
% 1"Smart Agriculture Systems: Soil Sensors and Plant Wearables for Smart and Precision Agriculture" — Extensive review, strong base for soil monitoring section. (MIT paper, highly relevant.)
% 2"Conception and Design of WSN Sensor Nodes Based on Soil Moisture Monitoring" — Focus on soil pH, turbidity, and temperature sensors. (Observation: Detailed sensor design.)
% 3"Applications of Sensor Networks and Remote Sensing in Precision Agriculture" — General but with some relevance to WSN in agriculture. (Observation: Somewhat shallow but covers multiple angles.)

O monitoramento do solo é uma ferramenta essencial para otimizar o crescimento das culturas, melhorar a eficiência produtiva e promover práticas agrícolas mais sustentáveis. Os sensores de solo permitem a medição contínua de parâmetros físicos e químicos, como umidade e concentração de nutrientes, fornecendo dados em tempo real para apoiar a tomada de decisão no campo. A demanda por essas tecnologias vem crescendo, impulsionada pelo aumento populacional, pela necessidade de ampliar a produção de alimentos e pela pressão por práticas agrícolas mais eficientes e conscientes em relação ao meio ambiente. Um exemplo representativo é o mercado de sensores de umidade do solo, que movimentou cerca de US\$ 147{,}5 milhões em 2020, com expectativa de atingir US\$ 360{,}9 milhões até 2027, refletindo o interesse global em tecnologias de agricultura digital \cite{yin_2021_smart} .

Historicamente, as recomendações de manejo agrícola foram desenvolvidas seguindo uma lógica de zonas agroecológicas amplas, como ocorreu durante a Revolução Verde, quando o foco estava apenas no aumento da produtividade por meio de fertilizantes sintéticos, sem considerar adequadamente as condições locais de solo e água, nem os impactos ambientais associados. Grande parte desse legado ainda persiste, com práticas baseadas em procedimentos centralizados e relações empíricas genéricas entre nutrientes, doses de fertilizantes e produtividade. Nesse contexto, os sensores de solo surgem como uma ferramenta fundamental para romper com esse modelo “top-down” e viabilizar uma abordagem “bottom-up”, em que as decisões de manejo passam a ser orientadas por dados reais, específicos de cada microambiente agrícola, no espaço e no tempo \cite{viscarrarossel_2016_soil}.

\subsection{SAW, RFID, e nanotechnology}
% monitoring system consists of RFID sensor, patrol car, farmland monitoring center and cloud platform
Como sabemos a agricultura pode cobrir vastas areas e nestes casos WSNs e tradiconalmente LoRaWAN são muito aplicáveis pois permitem cobrir estas áreas de forma remota \textcite{deng_2020_novel}. Dito isso, outras alternativas de cobertura para grandes areas tem sido exploradas além do mais tradicional LoRaWAN, o trabalho de \textcite{deng_2020_novel} apresenta um sistema de monitoramento que consiste em sensores RFID no solo, um veículo que se movimenta pela área monitorada coletando dados, e um centro de coleta e tratamento de dados, capaz de cobrir amplas áreas. Outro trabalho desenvolvido por ESSE CARA \textcite{akhileshnagpure_2022_water}, apresenta a utilização de drones de maneira similar a citada, para captar dados em áreas remotas.

O trabalho de \textcite{boada_2018_batteryless} apresenta o desenvolvimento de um sensor inovador de umidade do solo que opera sem bateria, utilizando tecnologia NFC com colheita de energia. O dispositivo é alimentado pelo campo magnético gerado pelo leitor NFC e realiza a medição de temperatura, umidade relativa e conteúdo volumétrico de água no solo. Um microcontrolador integrado processa os dados coletados e os transmite ao chip NFC via I2C, armazenando as informações no formato NDEF para posterior leitura. O estudo também compara diferentes métodos de medição de umidade do solo, selecionando a abordagem mais adequada às limitações energéticas do sistema. Com um princípio de funcionamento similar ao RFID, sensores passivos do tipo SAW também são apresentados como alternativas para monitoramento de características do solo, como explorado por BLABLABLA \textcite{akhileshnagpure_2022_water}. Uma forma inovadora de detecção de umidade do solo foi explorada por BLABLABLA \textcite{akhileshnagpure_2022_water} utilizando fibras ópticas instaladas em vastas áreas para verificar a concentração de água no solo.

AQUI UM PARAGRAFO SOBRE PLANTAS E NUTRIENTES, EXPLICAR SAW SENSOR E RFID?

\section{Monitoramento do Ar e Composição Química do Ar}
% Focus: Air quality, gas detection, environmental chemical monitoring.
% "SAW Sensors for Chemical Vapors and Gases" — Discusses SAW sensor principles for air/gas detection. (Observation: Emphasizes SAW advantages like sensitivity and robustness.)
% "A Progress Review on Solid‐State LiDAR and Nanophotonic Sensors" — Discusses LiDAR, photonic sensors applicable to particulate air detection or similar use cases.
Quando pensamos em ambientes urbanos integrados com IoT, dentro do conceito de smart cities, um dos parâmetros mais relevantes é a qualidade do ar, devido aos impactos diretos na saúde pública, no meio ambiente e na economia global. A poluição atmosférica em áreas urbanas, com distribuição espacial e temporal não uniforme, reforça a necessidade de sistemas de monitoramento com alta resolução espaço-temporal, algo que os sistemas tradicionais de monitoramento ainda não conseguem oferecer de forma escalável e com ampla cobertura de dados \cite{yi_2015_a}.

Nesse contexto, o avanço das tecnologias de sensores, como MEMS e redes de sensores sem fio (WSN), tem impulsionado o desenvolvimento do conceito de The Next Generation Air Pollution Monitoring System (TNGAPMS). Para esse tipo de aplicação, os gases de maior preocupação são monóxido de carbono (CO), dióxido de nitrogênio (NO$_2$), ozônio ao nível do solo (O$_3$) e dióxido de enxofre (SO$_2$). Atualmente, os sensores mais utilizados e adequados para o monitoramento de tais gases em cenários urbanos e industriais são os sensores eletroquímicos e os sensores de estado sólido (semicondutores), embora também existam outras tecnologias de baixo custo, como os sensores catalíticos, NDIR e PID, que são amplamente aplicados em diferentes contextos de detecção de gases \cite{yi_2015_a}.

\subsection{Detecção de Elementos Químicos}

O estudo de \textcite{devkota_2017_saw} investiga sensores SAW para detecção passiva de gases e vapores químicos no ar, baseados na interação dos compostos com a antena, que altera o sinal acústico recebido. Sensores SAW mostram-se viáveis para detectar gases inorgânicos como NH$_3$, NO$_2$, SO$_2$, CO, vapores orgânicos como tolueno, etanol, acetona e agentes químicos de guerra.
O trabalho destaca a importância da estabilidade dos materiais em ambientes extremos, o uso de antenas para operação sem fio e baixo consumo energético, e aponta para futuras pesquisas em materiais sensíveis, sensores flexíveis e arranjos multi-elemento para monitoramento simultâneo de vários gases.

FALAR DE OUTROS SENSORES MAIS COMUNS PARA ESTA APL

\subsection{Poluição e Qualidade do Ar}
O trabalho de \textcite{karagulian_2019_review} analisa o desempenho de sensores de baixo custo (LCS) para o monitoramento de diversos poluentes atmosféricos, incluindo monóxido de carbono (CO), óxidos de nitrogênio (NO e NO$_2$), ozônio (O$_3$) e material particulado (PM$_{2.5}$). A revisão destaca o potencial desses sensores para ampliar a cobertura espacial em áreas urbanas e remotas, além de avaliar diferentes métodos de calibração, como MLR, ANN, SVR e RF, levando em conta fatores como a umidade relativa, que influencia significativamente a medição de partículas.

O trabalho \textcite{yi_2015_a} apresenta uma revisão dos sistemas de monitoramento da poluição do ar baseados em redes de sensores sem fio (WSN), classificando-os em três categorias principais: Redes de Sensores Estáticas (SSN), Redes de Sensores Comunitárias (CSN) e Redes de Sensores Veiculares (VSN), de acordo com os tipos de portadores dos sensores. A análise destaca que muitas soluções atuais já demonstram ser viáveis em termos de resolução espaço-temporal, custo, eficiência energética, facilidade de implantação, manutenção e acesso público aos dados. No entanto, permanecem desafios como a falta de aquisição de dados tridimensionais, limitações na capacidade de monitoramento ativo e o uso de portadores não controlados ou parcialmente controlados, aspectos que devem ser aprimorados nas futuras gerações de sistemas de monitoramento da poluição do ar.
