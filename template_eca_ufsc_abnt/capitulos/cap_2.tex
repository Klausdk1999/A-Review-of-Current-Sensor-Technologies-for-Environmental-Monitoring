\chapter{Literature Review}

% 2 Sensors and Technologies for Environmental Monitoring
% Sensor classification from that “sensors daily use…” - Active or passive and applications.
% 2.1 Water Monitoring
% Focus: Water level (ultrasonic, LiDAR, float sensors),Water quality (pH, turbidity, EC sensors), flood detection, river monitoring.
% Innovations: Archimedes-based sensors, object recognition methods, SAW?.

% Relevant Articles:
% 1"An Intelligent Water Level Estimation System" — Uses object recognition via deep learning for water levels. (Observations: Estimation via camera + ML.)
% 2"A Technical Evaluation of Lidar-Based Measurement of River Water Levels" — Discusses LiDAR for precise water monitoring. (High relevance.)
% 3"A low-cost ultrasonic sensor for online monitoring of liquid level" — Simple ultrasonic-based approach. (Low relevance but practical example.)
% 4"SAW Sensor based a Novel Hydrostatic Liquid Level Sensor" — Novel SAW-based liquid level sensor. (Observation: Hydrostatic pressure with SAW.)
% 5"Modeling and testing of a highly sensitive surface acoustic wave hydrostatic liquid level sensor" — Focus on SAW tech for water levels.
% 6"Development of liquid level measurement technologies: A review" — Industrial-focused review of level sensors.


% 2.2 Soil Monitoring
% Focus: Soil moisture, nutrients, agriculture precision.
% Use of SAW sensors, RFID-based systems, and nanotechnology.
% Relevant Articles:
% 1"Smart Agriculture Systems: Soil Sensors and Plant Wearables for Smart and Precision Agriculture" — Extensive review, strong base for soil monitoring section. (MIT paper, highly relevant.)
% 2"Conception and Design of WSN Sensor Nodes Based on Soil Moisture Monitoring" — Focus on soil pH, turbidity, and temperature sensors. (Observation: Detailed sensor design.)
% 3"Applications of Sensor Networks and Remote Sensing in Precision Agriculture" — General but with some relevance to WSN in agriculture. (Observation: Somewhat shallow but covers multiple angles.)

% 2.3 Air and Chemical Monitoring
% Focus: Air quality, gas detection, environmental chemical monitoring.
% "SAW Sensors for Chemical Vapors and Gases" — Discusses SAW sensor principles for air/gas detection. (Observation: Emphasizes SAW advantages like sensitivity and robustness.)
% "A Progress Review on Solid‐State LiDAR and Nanophotonic Sensors" — Discusses LiDAR, photonic sensors applicable to particulate air detection or similar use cases.

This study proposes the development of a water level monitoring system using LoRa-based WSNs and evaluating different non-contact measurement methods. The choice for a wireless and non-contact reading sensor node is the most recommended in the sensing application of unhealthy and harsh environments because it has independent processing and wireless signal transmission. They have a significant advantage over traditional wired sensors, which are not a cheap and viable option for this type of application \cite{bhuyan_2010_intelligent}.



--------- General LORA REVIEW MENTIONS ------  GENERAL FLOOD REVIEW MENTIONS ------ ?? maybe put on fundaments chapter

Previously ultrasonic was defined as viable option to monitor water levels in river and channels using the sensor model GY-Us42, with results indicating that the device's average error is below 3\% \cite{mohammadrezamasoudimoghaddam_2024_a}. Other work concludes that the ultrasonic sensor model HC-SR04 is a technically and economically viable alternative to monitor water levels \cite{ pereira_2022_evaluation}, and the same sensor model is also praised as a good option for education, citizen science, and research due to low cost \cite{bresnahan_2023_a}. 

In previous research, LiDAR was explored as a cost-effective sensor for measuring water leves from bridges, with lab and field tests indication a good accuracy with 0.1\% error, but significant variations caused by sensor temperature and water roughness \cite{paul_2020_a}. LiDAR sensors installed on river banks for flash flood monitoring were also tested with good results, and indication that different amounts of small suspended particles on water could impact the results, and that the sensor could also be used to monitor these suspended particles \cite{tamari_2016_flash}. Another study explored how the LiDAR sensor model TF-mini compared to linigraph pressure sensors, showing the benefits of a non-contact measurement method of the LiDAR compared to a contact one with the linigraph and validating the LiDAR as an excellent choice among fluid level measurement technologies compared \cite{santana_2024_development}.

The contribution of this work is exploring more of the two different methods of non-contact water level measurement, comparing and definig best applications for each one. Adding to it, this work will also explore creating a low-cost, low-power, and long-range water level monitoring system using LoRaWAN technology. The system will be designed to operate in challenging environments, such as flood-prone river systems, where other more traditional monitoring methods may be impractical or too expensive.
