% ----------------------------------------------------------
\chapter{Conclusion}\label{cap:conclusao}
% ----------------------------------------------------------

% 5 Conclusion and Future Perspectives
% WSN for environmental monitoring continues to evolve rapidly.
% AI Increases the need for data, which can be supplied from sensors etc.
% Opportunities in integrating AI, improving sensor robustness, and extending network lifetime.
% The importance of standardizing data communication protocols (LoRa, LPWAN) and improving cybersecurity.
% Smart cities?
% IA on chip (edge) 


A proposta deste projeto foi desenvolver um sistema para automação de pedido e compras em pontos de venda, que permita integração com agentes de distribuição, de forma a automatizar todo o processo.

Com base nos resultados apresentados, pode-se concluir que o desenvolvimento do sistema obteve sucesso. A aplicação web desenvolvida permite que os usuários possam fazer seus pedidos, enquanto os funcionários dos pontos de venda têm acesso a informações precisas sobre os pedidos realizados, podendo acompanhar o status da entrega em tempo real. Além disso, o servidor de \textit{sockets} permite que os pontos de venda possam notificar os agentes de distribuição de qualquer atualização no status dos pedidos, de forma rápida e eficiente.

Por fim, a utilização de serviços de hospedagem em nuvem garantiu a disponibilidade e a escalabilidade do sistema, permitindo que ele possa ser usado por um grande número de usuários ao mesmo tempo.

Dessa forma, pode-se afirmar que o sistema de gerenciamento de pedidos para pontos de venda desenvolvido neste trabalho é uma solução eficiente e viável para empresas que desejam aprimorar seus processos de gerenciamento de pedidos e melhorar a experiência dos seus clientes, usando automação e tecnologias web modernas.

Para trabalhos futuros deseja-se incluir melhorias na interface, principalmente para exibir mais informações sobre os pedidos, e de forma geral torna-la mais amigável. Além disso, outro ponto que pode ser trabalhado seria o desenvolvimento de agentes de distribuição, conectando o NodeMCU com uma máquina de café, por exemplo, e distribuindo realmente os produtos.