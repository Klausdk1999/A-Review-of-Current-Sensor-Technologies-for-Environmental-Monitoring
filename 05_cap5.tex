% ----------------------------------------------------------------------- %
% Arquivo: cap.tex
% ----------------------------------------------------------------------- %
\chapter{Resultados}
\label{c_cap5}

Este capítulo é dedicado à apresentação e à discussão de resultados experimentais de modo a permitir que se possa avaliar a contribuição do trabalho, o alcance dos seus objetivos e a verificação das hipóteses de pesquisa.

Os experimentos realizados devem seguir um método científico (o qual deve ser descrito) e os seus resultados devem ser apresentados e discutidos pelo autor. Usualmente, utilizam-se tabelas, quadros e gráficos para apresentar os resultados experimentais, sendo que análise é feita de forma textual, conduzindo a discussão à verificação das hipóteses estabelecidas no início da pesquisa.




%_________________________
\section{Resultado dos Experimentos}
\label{s_c5_experimentos}

Esta seção descreve os resultados obtidos a partir de tal e tal...




%_________________________
\subsection{Experimento 01}
\label{ss_c5_CT01}

Os detalhes deste experimento são apresentados no \autoref{q_c5_CT01} de exemplo, que segue abaixo:

\


\begin{quadro}[!htbp]
 \caption{Caso de Teste 01 - exemplo do CT 01}
 \label{q_c5_CT01} 
 \begin{center}
 \begin{footnotesize} 

\begin{tabular}{|p{3.2cm}|p{13cm}|}
\hline
Tipo de teste & Funcional \\ \hline
Objetivo & texto \\ \hline
Pré-requisitos & texto \\ \hline
Dados de entrada & texto \\ \hline
Resultados esperados & texto \\ \hline
\end{tabular}
 
 \end{footnotesize}
 \end{center} 
 \raggedright Fonte: Elaborado pelo próprio autor. 
\end{quadro}


\noindent \textbf{Descrição do experimento 01:} aqui está toda a descrição do experimento...



%_________________________
\subsection{Experimento 02}
\label{ss_c5_CT02}

Descrição do experimento 02.




%_________________________
\subsection{Análise dos Experimentos}
\label{ss_c5_analise-experimentos}

Conclusão da análise feita...



%_________________________
\section{Avaliação da pesquisa de satisfação}
\label{s_c5_pesquisa}

Esta seção apresenta a avaliação da pesquisa de satisfação, realizada em um período de dez (10) dias, para o protótipo XYZ desenvolvido. O \autoref{apendice_f} apresenta o formulário de satisfação...., e o \autoref{apendice_g} apresenta as respostas coletadas...

A pesquisa foi respondida por vinte (20) profissionais de TI (tecnologia da informação) que trabalham em instituições governamentais e por treze (13) profissionais de TI que trabalham para empresas privadas, somando trinta e três (33) avaliações. 

%_________________________
\subsection{Avaliação dos Resultados}
\label{ss_c5_avaliacao-pesquisa}


\begin{enumerate}
    %Pesquisa de satisfação - experimentação 01
    \item Identificar tal e tal coisa.
    
    Descrição das análise...
\end{enumerate}
    
    
%_________________________
\section{Comparação com os trabalhos relacionados}
\label{s_c5_trab-relacionados}

Este trabalho, apresentou uma proposta XYZ, etc, etc, etc...


%_________________________
\section{Considerações}
\label{s_c5_consideracoes}

Este capítulo pode ter uma última seção como esta denominada ``Considerações'' ou ``Discussão'' consolidando a análise dos resultados.